%%! define('NOTE_STYLE', 'footnote');
\documentclass[a4paper]{article}

\usepackage[font=opentype,pinyin=math,href=yes,titlesec=yes]{xepkgs}
\usepackage[left=2cm,right=2cm,top=2cm,bottom=2cm,dvipdfm]{geometry}
\usepackage{indentfirst}

\titleformat{\section}{\centering\heiti\huge}{}{1em}{}
\setlength{\parskip}{1.5\baselineskip}

\newcommand{\comment}[1]{{\noindent\kaiti\large #1}}
%%{
if (NOTE_STYLE == 'gezhu')
    $note_spec = "gezhu,color=darkgray,font=fangti";
else
    $note_spec = "footnote,color=darkgray,font=fangti,size=normalsize,raise=0.5ex";
echo "\\newcommand{\\zyparx}[1]{{\\vspace{-3ex}\\zypar*[pysize=normalsize,pycolorx=red,ccsize=huge]{{$note_spec}}{#1}}}\n";
%%}

\usepackage{footmisc}

\renewcommand{\thefootnote}{\normalsize[\arabic{footnote}] }

\begin{document}

\section[]{醉翁亭记(\fangti 欧阳修)\zynotemark\zynotetext{\fangti\normalsize\color{darkgray}选自《欧阳文忠公文集》。欧阳修(1007年-1072年),字永叔,号醉翁、六一居士,汉族,吉州永丰(今江西省吉安市永丰)人,北宋政治家、文学家。}}

%%<<<*paras
%%! foreach ($paras as $par) {
%%!     echo "\zyparx{%\n";
%%!     if (NOTE_STYLE == 'gezhu')
%%!         $par = str_replace("。}", "}", $par);
%%!     print_zy($par, false, $note_spec);
%%!     echo "\n}\n";
%%!     echo "\n";
%%! }

环滁皆山也@{环滁皆山也:环,环绕。滁:滁州,今安徽省滁州市琅琊区。皆:副词,都。}。
其西南诸峰,
林壑尤美@{林壑尤美:壑,山谷。尤,格外,特别。},
望之蔚然而深秀者,琅琊也@{蔚然:草木茂盛的样子。琅琊:琅琊山,在滁州西南约5公里。}。
山@{山:名词作状语,沿着山路。}行六七里,
渐闻水声潺潺而泻出于两峰之间者@{潺潺:流水声。而:助词,表示动作承接。},
酿泉@{酿泉:泉水名}也。
峰回路转@{峰回路转:山势回环,路也跟着拐弯。回:回环,曲折环绕。},
有亭翼然临于泉上者@{翼然:像鸟张开翅膀一样。然:...的样子。临:靠近。于:在。},
醉翁亭也。
作@{作:建造。}亭者谁?
山之僧智仙也。
名@{名:名词作动词,命名。}之者谁?
太守自谓@{自谓:自称,用自己的别号来命名。}也。
太守与客来饮于此,
饮少辄醉@{饮少辄醉:喝一点儿酒就醉了。少:一点。辄:就},
而年又最高@{年又最高:年纪又是最大的。},
故自号曰@{号:名词作动词,取别号。曰:叫做。}醉翁也。
醉翁之意不在酒@{意:这里指情趣。““醉翁之意不在酒”,后来用以比喻本意不在此而另有目的。},
在乎@{在乎:在于。}山水之间也。
山水之乐,
得之心而寓之酒也@{得:领会。寓:寄托}。

若夫(fu2)日出而林霏开@{林霏:树林里的雾气。霏,原指雨、雾纷飞,此处指雾气。开:消散,散开。},
云归而岩穴暝@{归:聚拢,指散开的云又回聚到山来。暝:昏暗。},
晦明变化者@{晦:阴暗。晦明:指天气阴晴明暗。},
山间之朝(zhao1)暮也。
野芳发而幽香@{芳:香花。发:开放。},
佳木秀而繁阴@{佳木:好的树木。秀:植物开花、结实。繁阴:一片浓密的树荫。},
风霜高洁@{风霜高洁:就是风高霜洁。天高气爽,霜色洁白。},
水落而石出者,
山间之四时@{四时:指春夏秋冬四季。}也。
朝(zhao1)而往,
暮而归,
四时之景不同,
而乐亦无穷也。

至于负者歌于途@{至于:连词,于句首,表示两段的过渡,提起另事。负者:背东西的人。},
行者休于树@{休于树:在树下休息。},
前者呼,
后者应(ying4),
伛偻(lv3)提携@{伛偻:腰背弯曲的样子,这里指老年人。提携:搀扶,引领(着小孩子)。全句指老人弯着腰拉着小孩子。},
往来而不绝者,
滁人游也。
临溪而渔@{临:靠近,这里是“……旁”的意思。渔:捕鱼。},
溪深而鱼肥,
酿泉@{酿泉:名作状,用泉水。}为酒,
泉香而酒洌@{洌:清醇。},
山肴野蔌@{山肴:野味。野蔌:野菜。蔌,菜蔬。},
杂然而前陈者@{杂然:众多而杂乱的样子。陈:摆列。},
太守宴也。
宴酣@{酣:尽情地喝酒。}之乐,
非丝非竹@{丝:琴、瑟之类的弦乐器。竹:箫、笛之类的管乐器。},
射@{射:这里指投壶,宴饮时的一种游戏,把箭向壶里投,投中多的为胜,负者照规定的杯数喝酒。}者中(zhong4),
弈@{弈:下棋。这里用做动词,下围棋。}者胜,
觥筹交错@{觥筹交错:酒杯和酒筹相错杂。形容喝酒尽欢的样子。觥:酒杯。筹:酒筹,宴会上行令或游戏时饮酒计数用的签子。},
起坐而喧哗者,
众宾欢也。
苍颜@{苍颜:脸色苍老。}白发(fa4),
颓然乎其间者@{颓然乎其间:醉醺醺地坐在众人中间。颓然,原意是精神不振的样子,这里形容醉态。},
太守醉也。

已而@{已而:不久。}夕阳在山,
人影散(san3)乱,
太守归而宾客从也。
树林阴翳@{阴翳:形容枝叶茂密遮盖成阴。翳:遮蔽。},
鸣声上下@{鸣声上下:意思是鸟到处叫。上下,指高处和低处的树林。},
游人去而禽鸟乐也。
然而禽鸟知山林之乐,
而不知人之乐;
人知从太守游而乐,
而不知太守之乐其乐@{乐(1)其乐(2):以游人的快乐为快乐。乐1:意动用法,以…为乐。乐2:快乐。}也。
醉能同其乐,
醒能述以文者@{醉能同其乐,醒能述以文者:醉了能够同大家一起欢乐,醒了能够用文章记述这乐事的人。},
太守也。
太守谓@{谓:为,是。}谁?
庐陵@{庐陵:庐陵郡,就是吉洲。现在江西省吉安市,欧阳修的家乡。}欧阳修也。

%%>>>

\end{document}
