%%! define('NOTE_STYLE', 'footnote');
\documentclass[a4paper]{article}

\usepackage[font=opentype,pinyin=math,href=yes,titlesec=yes]{xepkgs}
\usepackage[left=2cm,right=2cm,top=2cm,bottom=2cm,dvipdfm]{geometry}
\usepackage{indentfirst}

\titleformat{\section}{\centering\heiti\huge}{}{1em}{}
\setlength{\parskip}{1.5\baselineskip}

\newcommand{\comment}[1]{{\noindent\kaiti\large #1}}
%%{
if (NOTE_STYLE == 'gezhu')
    $note_spec = "gezhu,color=darkgray,font=fangti";
else
    $note_spec = "footnote,color=darkgray,font=fangti,size=normalsize,raise=0.5ex";
echo "\\newcommand{\\zyparx}[1]{{\\vspace{-3ex}\\zypar*[pysize=normalsize,pycolorx=red,ccsize=huge]{{$note_spec}}{#1}}}\n";
%%}

\usepackage{footmisc}

\renewcommand{\thefootnote}{\normalsize[\arabic{footnote}] }

\begin{document}

\section[]{岳阳楼记(\fangti 范仲淹)\zynotemark\zynotetext{\fangti\normalsize\color{darkgray}选自《范文正公集》。范仲淹(989-1052),字希文,死后谥号文正,世称范文正公,苏州吴县(现江苏省吴县)人,北宋政治家、军事家、文学家。}}

%%<<<*paras
%%! foreach ($paras as $par) {
%%!     echo "\zyparx{%\n";
%%!     if (NOTE_STYLE == 'gezhu')
%%!         $par = str_replace("。}", "}", $par);
%%!     print_zy($par, false, $note_spec);
%%!     echo "\n}\n";
%%!     echo "\n";
%%! }

庆历四年@{庆历四年:公元1044年。庆历,宋仁宗赵祯的年号(1041-1048)。本文句末中的“时六年”,指庆历六年(1046),点名作文的时间。}春,
滕子京谪守巴陵郡@{滕子京谪守巴陵郡:滕子京降职任岳州太守。滕子京,名宗谅,字子京,范仲淹的朋友。谪,封建王朝官吏降职或远调。守,指做太守。巴陵:郡名,即岳州,治所在今湖南省岳阳市。}。
越@{越明年:到了第二年,就是庆历五年(1045)。越,经过。}明年,
政通人和@{政通人和:政事通顺,百姓和乐。政,政事;通,顺利;和,和乐。这是赞美滕子京的话。},
百废具兴@{百废具兴:各种该办而未办的事都兴办起来了。废,该办而未办的事。具,通“俱”,全、皆。兴,兴办。},
乃重(chong2)修岳阳楼,
增其旧制@{乃重修岳阳楼,增其旧制:乃,于是;就。增,扩大。旧制:原有的建筑规模。},
刻唐贤今人诗赋于其上,
属(zhu3)予(yu2)作文以记之@{属予作文以记之:属,同“嘱”,嘱托。作文,创作文章。以,用来}。

予(yu2)观夫(fu2)巴陵胜状@{予观夫巴陵胜状:夫,指示代词,相当于“那”。胜状,胜景,美好景色。},
在洞庭一湖。
衔远山,
吞长江,
浩浩汤(shang1)汤(shang1)@{衔远山,吞长江,浩浩汤汤:衔,衔接。吞,吞纳。浩浩汤汤:水势浩大的样子。},
横无际涯@{横无际涯:宽阔无边。横:广远。涯,边。际涯:边际。(际、涯的区别:际专指陆地边界,涯专指水的边界)。};
朝(zhao1)晖夕阴,
气象万千@{朝晖夕阴,气象万千:或早或晚阴晴多变化,一天里气象变化多端。朝,在早晨,名词做状语。晖:日光。阴,阴暗。气象,景象。万千,千变万化。};
此则岳阳楼之大观也@{此则岳阳楼之大观也:此,这。则,就。大观,雄伟壮丽的景象。},
前人之述备矣@{前人之述备矣:前人的记述很详尽了。前人之述,指上面说的“唐贤今人诗赋”。备,详尽,完备。矣,语气词“了”。之,的。}。
然则北通巫峡@{然则北通巫峡:然则:(既然)这样那么,那么。北:名词用作状语,向北。},
南极潇湘@{南极潇湘:南面直达潇水、湘水。潇水是湘水的支流。湘水流入洞庭湖。南,向南。极,尽,到……尽头。},
迁客骚人,
多会于此@{迁客骚人,多会于此:迁客,被贬谪流迁的人。骚人,诗人。战国时屈原作《离骚》,因此后人也称诗人为骚人。会,聚会。于,在。此,这里。},
览物之情,
得无异乎@{览物之情,得无异乎:观赏自然景物时触发的情感,怎能不会有所不同呢?览,看,观赏。得无……乎,难道没有……吗?莫非……吧,大概……吧。异:不同。}?

若夫(fu2)霪雨霏霏@{若夫霪雨霏霏:若夫,用在一段话的开头引起论述的词。下文的“至若”用在又一段话的开头引起另一层论述。“若夫”近似“像那”。“至若”近似“至于”“又如”。霪雨,即“淫雨”,连绵不断的雨。霏霏,雨(或雪)繁密的样子。淫,过多。},
连月不开@{开:放晴。};
阴风怒号(hao2),
浊浪排空@{阴风怒号,浊浪排空:阴,阴冷。号,呼啸;浊,浑浊。排空,冲向天空。};
日星隐曜@{日星隐曜:太阳和星星隐藏起光辉。曜,光辉,光芒。},
山岳潜形@{山岳潜形:山岳隐没了形体。岳,高大的山。潜,潜藏。形,形迹。};
商旅不行,
樯倾楫摧@{樯倾楫摧:桅杆倒下,船桨折断。樯,桅杆。楫,桨。倾,倒下。};
薄(bo2)暮冥冥@{薄暮冥冥:傍晚天色昏暗。薄,迫近。冥冥:昏暗的样子。},
虎啸猿啼;
登斯@{斯:这,在这里指岳阳楼。}楼也,
则有去国怀乡,
忧谗畏讥@{则有去国怀乡,忧谗畏讥:则,就。有,产生……(的情感)。去国怀乡,忧谗畏讥:离开京都,怀念家乡,担心(人家)说坏话,惧怕(人家)批评指责。去,离开。国,京都。去国,离开京都,也即离开朝廷。畏,害怕,惧怕。忧,担忧。谗,谗言。讥,讥讽。},
满目萧然,
感极而悲者矣@{满目萧然,感极而悲者矣:萧然,萧条的样子。感,感慨。极,到极点。而,表示顺接。}!

至若春和景明@{至若春和景明:如果到了春天气候暖和,阳光明媚。春和,春风和煦。景,日光。明,明媚。},
波澜不惊@{波澜不惊:波澜平静。惊,起伏。这里有“起”、“动”的意思。},
上下天光,
一碧万顷@{上下天光,一碧万顷:上下天色湖光相接,一片碧绿,广阔无际。万顷,极言其广。};
沙鸥翔集,
锦鳞游泳@{沙鸥翔集,锦鳞游泳:沙鸥,沙洲上的鸥鸟。翔集,时而飞翔,时而停歇。集,栖止,鸟停息在树上。锦鳞,指美丽的鱼。鳞,代指鱼。游:指水面浮行。泳,指水中潜行。},
岸芷汀兰@{岸芷汀兰:岸上的香草与小洲上的兰花(此句为互文)。芷:香草的一种。汀:水边平地。},
郁郁@{郁郁:形容草木茂盛。}青青。
而或长烟一空@{而或长烟一空:有时大片烟雾完全消散。而或,有时。长:大片。一,全。空:消散。},
皓月千里@{皓月千里:皎洁的月光照耀千里。},
浮光跃金@{浮光跃金:波动的光闪着金色。这是描写月光照耀下的水波。},
静影沉璧@{静影沉璧:湖水平静时,明月映入水中,好似沉下一块玉壁。璧,圆形的玉。},
渔歌互答@{渔歌互答:渔人唱着歌互相应答。答,应和。},
此乐何极@{何极:哪里有尽头。极:尽头。}!
登斯楼也,
则有心旷神怡@{心旷神怡:心情开朗,精神愉快。旷,开阔。怡,愉快。},
宠辱偕忘@{宠辱偕忘:荣耀和屈辱都忘了。偕,一起。宠,荣耀。}、
把酒临风@{把酒临风:端酒当着风,就是在清风吹拂中端起酒来喝。把,拿。临,面对。},
其喜洋洋@{洋洋:高兴得意的样子。}者矣!

嗟夫(fu2)@{嗟夫:唉。嗟夫为两个词,皆为语气词。}!
予(yu2)尝求古仁人之心@{予尝求古仁人之心:尝,曾经。求,探求。古仁人,古时品德高尚的人。心,思想感情。},
或异二者之为@{或异二者之为:或许和以上两种人的思想感情有所不同。或,近于“或许”“也许”的意思,表委婉口气。异,不同于。为,心理活动。二者,这里指前两段的“悲”与“喜”。},
何哉?
不以物喜,
不以己悲@{不以物喜,不以己悲:不因为外物(好坏)和自己(得失)而或喜或悲(此句为互文)。以,因为。},
居庙堂之高,
则忧其民@{居庙堂之高则忧其民:在朝中做官担忧百姓。意为在朝中做官。庙,宗庙。堂,殿堂。庙堂:指朝廷。下文的“进”,即指“居庙堂之高”。};
处(chu3)江湖之远,
则忧其君@{处江湖之远则忧其君:处在僻远的地方做官则为君主担忧,意思是远离朝廷做官。下文的“退”,即指“处江湖之远”。之:定语后置的标志。}。
是进亦忧,
退亦忧@{是:这样。进:在朝廷做官。退:不在朝廷做官。};
然则何时而乐耶?
其必曰:“先天下之忧而忧,后天下之乐而乐”乎@{其必曰“先天下之忧而忧,后天下之乐而乐”乎:他们一定要说“在天下人担忧之前先担忧,在天下人享乐之后才享乐”吧。先,在……之前;后,在……之后。其:指“古仁人”。必:一定。}。
噫!
微斯人,
吾谁与归@{微斯人,吾谁与归:(如果)没有这种人,我同谁一道呢?微,没有。斯人,这样的人。谁与归,就是“与谁归”。归,归依。}!
时六年九月十五日。

%%>>>

\end{document}
