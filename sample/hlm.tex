\documentclass{article}

\usepackage[font=opentype,pinyin=math,href=yes,titlesec=yes]{xepkgs}
\usepackage[left=2cm,right=2cm,top=2cm,bottom=2cm,dvipdfm]{geometry}
\usepackage{indentfirst}
\usepackage{longtable}

\titleformat{\section}{\heiti\Large}{\CJKnumber{\thesection}}{1em}{}
\newenvironment{lpar}{\vspace{-1ex}\begin{longtable}{lp{0.85\columnwidth}}}{\end{longtable}}
\newcommand\lver[2]{\vspace{1ex}#1&#2\\}

\begin{document}

\begin{center}\heiti\LARGE 越剧《红楼梦》唱词\end{center}
\kaiti\large

\ 

\setlength\LTleft\parindent
%\section{主题曲}
%
%\begin{lpar}
%\lver{休笑前人痴,由来同一梦。 }{}
%\lver{绣金翠袖,难揾悲金悼玉泪。}{}
%\lver{菱花镜里,谁拥旷世情种。 }{}
%\lver{罗带同心结未成,鹊桥长恨无归路。 }{}
%\lver{红楼今犹在,唯有风月鉴空。}{}
%\end{lpar}

\section{黛玉进府}

\begin{lpar}
\lver{紫 鹃}{啊呀,林姑娘来了。啊呀你们快来看呀,林姑娘来了……}
\lver{众 人}{啊,老太太……林姑娘来了……}
\lver{内 声}{林姑娘走好,林姑娘走好……}
\lver{合 唱}{乳燕离却旧时窠,孤女投奔外祖母。}
\lver{林黛玉}{外祖母家确与别家不同。}
\lver{合 唱}{记住了不可多说一句话,不可多走一步路。}
\lver{贾 母}{啊,我的外孙女来了,在哪里呀,在哪里呀?外孙女儿!}
\lver{林黛玉}{外祖母。}
\lver{贾 母}{外孙女儿。}
\lver{林黛玉}{外祖母……}
\lver{贾 母}{我的心肝宝贝呀……\color{red}可怜你年幼失亲娘,孤苦伶仃实堪伤。又无兄弟共姐妹,似一枝寒梅独自放。今日里接来娇花倚松栽,从今后,在白头外婆怀里藏。}
\lver{王夫人}{是啊。}
\lver{贾 母}{这是你的二舅母,快过去见过。}
\lver{林黛玉}{是。拜见二舅母。}
\lver{王夫人}{啊呀,不消了,外甥女儿,快起来吧,来,这旁坐下。}
\lver{王熙凤}{啊,林姑娘来了吗,真的来了啊。在哪里呀?啊呀,哈哈哈哈。我来迟了,来迟了,未曾迎接远客。啊呀,老祖宗,我来迟了。\color{red}昨日楼头喜鹊噪,今朝庭前贵客到。}
\lver{贾 母}{你呀,不认识她。她是我们这里有名的一个“泼辣货”,南京人所谓“辣子”,你只要叫她一声“凤辣子”就是了。}
\lver{王夫人}{她就是你琏二嫂子,学名唤做王熙凤。}
\lver{林黛玉}{拜见二嫂子。}
\lver{王熙凤}{噢,起来,起来起来。啊呀,好一个妹妹。\color{red}休怪我一双凤眼痴痴瞧,似这般美丽的人儿天下少。哪像个老祖宗膝前的外孙女,分明是玉天仙离了蓬莱岛。怪不得我家的老祖宗,在人前背后常夸耀。唉,只是我妹妹好命苦,姑妈偏就去世早。}
\lver{贾 母}{嗳,}
\lver{王夫人}{快不要伤心。}
\lver{贾 母}{我一天愁云方才消,你何必又招我烦恼。}
\lver{王熙凤}{哎呀真是,我一见了妹妹,一心都在她身上,又是欢喜,又是伤心,竟忘了老祖宗了。老祖宗,喏,该打,该打,该打!}
\lver{众 人}{哈哈哈……}
\lver{王熙凤}{噢,妹妹,你如今来到这里呀,\color{red}休当作粉蝶儿寄居在花丛,这家中就是你家中。你要吃要用把嘴唇动,受委屈告诉我王熙凤。}
\lver{林黛玉}{多谢嫂子费心。}
\lver{内 声}{宝二爷回来啦,啊呀,宝二爷回来了……}
\lver{贾宝玉}{老祖宗安。}
\lver{贾 母}{起来。}
\lver{贾宝玉}{太太安。}
\lver{王夫人}{起来。}
\lver{贾 母}{宝玉啊,}
\lver{贾宝玉}{啊?}
\lver{贾 母}{家里来了客人,怎么不过来见过林妹妹?}
\lver{王夫人}{是啊,快过去见过林妹妹。}
\lver{贾宝玉}{林妹妹。\color{red}天上掉下个林妹妹,似一朵轻云刚出岫。}
\lver{林黛玉}{\color{red}只道他腹内草莽人轻浮,却原来骨格清奇非俗流。}
\lver{贾宝玉}{\color{red}闲静犹似花照水,行动好比风拂柳。}
\lver{林黛玉}{\color{red}眉梢眼角藏秀气,声音笑貌露温柔。}
\lver{贾宝玉}{{\color{red}眼前分明外来客,心底却似旧时友。}哎,这个妹妹,我好像曾看见过的。}
\lver{贾 母}{啊呀,又要胡说了,你何曾见过啊?}
\lver{贾宝玉}{虽没见过,看见面善,心里好像认识的一般呢。}
\lver{贾 母}{好啊,哈,这样么以后在一起就和睦了。坐下坐下。}
\lver{贾宝玉}{妹妹,你读过书吗?}
\lver{林黛玉}{只上过一年学。}
\lver{贾宝玉}{噢,尊名?}
\lver{林黛玉}{名唤黛玉。}
\lver{贾宝玉}{表字呢?}
\lver{林黛玉}{无字。}
\lver{贾宝玉}{无字?好,我送妹妹一字,莫若“颦颦”两字极妙啊。}
\lver{王熙凤}{哎,什么叫“颦颦”啊?}
\lver{贾宝玉}{《古今人物通考》上说:“西方有石名黛,可做画眉之墨。”看,妹妹眉尖若蹙,取这个字岂非甚美呀?}
\lver{王熙凤}{只怕又是杜撰的。}
\lver{贾宝玉}{嗳,除了四书五经,杜撰的也太多了。妹妹,你有玉没有呀?}
\lver{林黛玉}{我没有玉。你那块玉也是件稀罕之物,岂能人人都有?}
\lver{贾宝玉}{什么稀罕的东西,人的高下不识,还说灵不灵呢。我可不要这东西。}
\lver{王夫人}{宝玉,你……}
\lver{贾 母}{宝玉,孽障!你生气,要打骂人容易,何苦去摔你那命根子哦!}
\lver{贾宝玉}{家里姐姐妹妹都没有,只有我有,我说没趣嘛。今天来了神仙似的妹妹,她也没有,可见这不是好东西!}
\lver{王夫人}{宝玉啊,快戴上。}
\lver{贾 母}{我不要。}
\lver{王夫人}{宝玉,宝玉。}
\lver{贾 母}{宝玉!}
\lver{王熙凤}{宝兄弟,戴上去,啊。}
\lver{王夫人}{宝玉,当心你爹知道,快戴上。}
\lver{王熙凤}{啊呀,宝兄弟啊,老祖宗不是常常说的吗,这宝贵家业就指望着你这个命根子呢!}
\lver{贾 母}{对呀!}
\end{lpar}

\section{识金锁}

\begin{lpar}
\lver{贾宝玉}{宝姐姐请啊。}
\lver{薛宝钗}{宝兄弟请啊。\color{red}老太太跟前请罢安,把林家妹妹来探望。}
\lver{贾宝玉}{林妹妹,林妹妹,宝姐姐来看你了呀……\color{red}这正是上庙不见土地神,}
\lver{薛宝钗}{\color{red}略等片刻又何妨。}
\lver{贾宝玉}{宝姐姐请坐啊。}
\lver{薛宝钗}{咦,宝兄弟,你颈上挂的那块玉,我虽然听见过,却未曾细细地赏鉴过,今天倒要见识一下。}
\lver{贾宝玉}{好啊。}
\lver{薛宝钗}{“莫失莫忘,仙寿恒昌”,“莫失莫忘,仙寿恒昌”……咦,你看得发呆作什么?}
\lver{莺 儿}{我听这两句话倒像姑娘金锁上的两句话是一对呢。}
\lver{贾宝玉}{宝姐姐,你金锁上面也有几个字,今天也让我赏鉴赏鉴。}
\lver{薛宝钗}{你不要听她讲,没有什么字的。}
\lver{贾宝玉}{宝姐姐,你看了我的,岂可不让人家看你的呢?}
\lver{薛宝钗}{我这个没有什么好看的呀。}
\lver{贾宝玉}{让我看看呀,看看呀。}
\lver{薛宝钗}{喏,只是上面有两句吉利话罢了。}
\lver{贾宝玉}{“不离不弃,芳龄永继”,呀,这两句话,真像和我是一对喏。好香啊,宝姐姐,你的衣服熏得好香啊。}
\lver{薛宝钗}{我最怕熏香,好好的衣裳为什么要熏呢?}
\lver{贾宝玉}{那是什么香呢?}
\lver{薛宝钗}{噢,想是我早晨起来,吃的“冷香丸”的香气吧。}
\lver{贾宝玉}{冷香丸?那么好闻,给我一丸尝尝吧。}
\lver{薛宝钗}{你又混闹了,药怎么可以瞎吃的呀?}
\lver{贾宝玉}{宝姐姐,给我一丸尝尝。}
\lver{薛宝钗}{你又混闹了,药怎么可以瞎吃的呢!}
\lver{贾宝玉}{这香气好闻呀。}
\lver{薛宝钗}{咦,怎么林妹妹到这般时候还没有回来,宝兄弟,我不等她了,改日再来望她吧。}
\lver{贾宝玉}{好吧,我送宝姐姐出去,改日再来吧。}
\lver{薛宝钗}{请。}
\lver{贾宝玉}{请吧。呀,林妹妹,方才宝姐姐来看你,你上哪里去了呀?怎么,睡着了?早饭刚刚吃好又要睡觉了。}
\lver{林黛玉}{你吵我做什么?}
\lver{贾宝玉}{好妹妹你看呀,{\color{red}春光如锦不去赏,合起眼皮入睡乡,饭后贪睡易积食,}来,\color{red}我替你解闷去寻欢畅。}
\lver{林黛玉}{你到别处去玩吧。}
\lver{贾宝玉}{我上哪里去呀?我看见他们怪腻的。}
\lver{林黛玉}{好,你既愿在这里,\color{red}就老老实实端正坐,休像蜜糖黏在人身上。}
\lver{贾宝玉}{好,我也躺着。}
\lver{林黛玉}{你躺着吧。}
\lver{贾宝玉}{没有枕头啊,好,我们两个合一个枕头吧。}
\lver{林黛玉}{放屁,外面屋子里有的是枕头。}
\lver{贾宝玉}{我不要嘛,外面的枕头都是肮脏老婆子们用的呀。}
\lver{林黛玉}{你啊,真是我命中的魔星,喏。咦,你脸上又是给谁的指甲划破了?}
\lver{贾宝玉}{不是的,方才搽了点胭脂膏。}
\lver{林黛玉}{你又在做这些事了,要是传到舅舅耳朵里去,大家都不得安心了。}
\lver{贾宝玉}{嗯,好香啊,这香气奇怪呀,又不是香饼子的香,又不是香袋儿的香,哎,这是什么香啊?}
\lver{林黛玉}{难道我有什么奇香不成?好,就算我有奇香,我问你,你有暖香没有?}
\lver{贾宝玉}{什么暖香?}
\lver{林黛玉}{啊,蠢材啊蠢材,你有“玉”,人家就有“金”来配你,人家有“冷香”你就没有“暖香”去配她?}
\lver{贾宝玉}{好,我说了一句话,你拉上了这么许多,我今天不给你个厉害喏……}
\lver{林黛玉}{哦,哈……}
\end{lpar}

\section{读西厢}

\begin{lpar}
\lver{贾宝玉}{谁?}
\lver{茗 烟}{哎,二爷,原来你在这里。把人家找得好苦啊。}
\lver{贾宝玉}{你到这里来来做什么?}
\lver{茗 烟}{二爷,你上次叫我弄的书,我给你拿来了。}
\lver{贾宝玉}{在哪里?}
\lver{茗 烟}{喏。}
\lver{贾宝玉}{《西厢记》?}
\lver{茗 烟}{二爷,对不对呀?}
\lver{贾宝玉}{对,对呀,对对对对对!}
\lver{茗 烟}{二爷,昨天你在老爷面前做的诗,人人都说你那些诗做的好,亏得老爷也喜欢。昨天你在老爷面前得了彩头,今天也该赏赏我啦。}
\lver{贾宝玉}{好,明天给你一吊钱。}
\lver{茗 烟}{谁没见过一吊钱啊,二爷,把这个象牙雕刻的,赏给我吧?}
\lver{贾宝玉}{好,拿去拿去。}
\lver{茗 烟}{把这只荷包也送给我吧?}
\lver{贾宝玉}{唉,这荷包是林妹妹送给我的呀,谁也不须拿。}
\lver{茗 烟}{二爷。}
\lver{贾宝玉}{喏,这个你拿去吧。去去去去去。}
\lver{贾宝玉}{这样的好书,老爷却不许我读,我今天背地里\color{red}偏要读它个爽快。读遍书斋经与史,难得《西厢》绝妙词。羡张生,琴心能使莺莺解,慕莺莺,深情更比张生痴。叹宝玉,身不由己圈在此,但愿得今晚梦游普救寺。}
\lver{林黛玉}{咦!}
\lver{贾宝玉}{哦,是你呀。}
\lver{林黛玉}{你在这里做什么?噢,原来躲在这里用功,这样一来呀,可要蟾宫折桂了呢?}
\lver{贾宝玉}{你取笑我做什么?你不是不知道,我最讨厌诓功名、混饭吃的八股文章,你还提这些!}
\lver{林黛玉}{不是那些书,那又是什么书呢?不要在我面前弄鬼了,趁早给我看看。}
\lver{贾宝玉}{好妹妹,我在这里看这个书,除了花鸟知道,没有一个人知道。给你看我是不怕的,哎,你好歹不要告诉人。}
\lver{林黛玉}{什么书啊?}
\lver{贾宝玉}{真正是好文章!你要是看了,连饭也不想吃呢。}
\lver{林黛玉}{《西厢记》!}
\lver{贾宝玉}{嘘!轻一点。好妹妹,真正是好文章。你看,你说好不好?}
\lver{林黛玉}{嗯。}
\lver{贾宝玉}{啊,妹妹,\color{red}我是个多愁多病身啊,你就是那倾国倾城的貌!}
\lver{林黛玉}{啊呀,你该死的,\color{red}胡说八道,弄出这淫词艳曲来调笑,混帐的话儿欺侮人,我可要到舅舅跟前将你告。}
\lver{贾宝玉}{好妹妹,{\color{red}我无非过目成诵顺口念,好妹妹你千万饶我这一遭。我若有心欺侮你,}好,明朝让我跌在这池子里,让癞头鼋\color{red}把我吞吃掉。}
\lver{林黛玉}{\color{red}那张生,一封书敢于退贼寇;那莺莺,八行笺人约黄昏后;那红娘,三寸舌降服老夫人;那惠明,五千兵馅做肉馒头。我以为你也胆如斗,呸,原来是个银样蜡枪头。}
\lver{贾宝玉}{好!你说说,你这个呢,我也要去告诉去。}
\lver{林黛玉}{你说你能过目成诵,难道我就不能一目十行么?你去呀,你去呀。}
\lver{贾宝玉}{好妹妹,我们不谈这些了好吗?我们坐下来谈别的吧。好妹妹,我上次到你房里来,看见你又在做针线,你做个香袋送给我好不好?}
\lver{林黛玉}{那要看我高兴不高兴。}
\lver{贾宝玉}{你送我个香袋,我也送你个好东西。这是北静王送给我的,是皇上赐下来的呀。}
\lver{林黛玉}{什么臭男人拿过的,我不要这东西!}
\lver{贾宝玉}{你不要这东西,我可要你的香袋。要,我要香袋,哎,香袋。}
\lver{林黛玉}{你要一个香袋,那很容易,横竖今后有人会替你做了,人家比我又会念,又会做,又会说会笑……}
\lver{贾宝玉}{你又来了。你这个人哪,难道连“亲不隔疏,后不僭先”都不知道?那第一件,我们两人是姑舅亲,宝姐姐是两姨亲,论亲戚也比你远;第二件,你先来,我们两人一桌吃,一床睡,从小一起长大,她是才来的呀,岂可为了她而疏远你的呢?}
\lver{林黛玉}{啐!我难道叫你去疏远她,那我成了什么人了呢?我为的是我的心。}
\lver{贾宝玉}{我也为的是我的心,难道只知道你的心,而不知道我的心不成?}
\lver{林黛玉}{今天天气分明冷了一些,你穿得这样单薄,回头冷了怕又要伤风。}
\lver{贾宝玉}{看,看你自己也穿得那么单薄。}
\end{lpar}

\section{不肖种种}

\begin{lpar}
\lver{贾宝玉}{每日里送往迎来把客陪,焚香叩头祭祖先。垂手恭敬听教诲,味同嚼蜡读圣贤。这饵名钓禄的臭文章,读得我头昏目眩实可厌。}
\lver{晴 雯}{眉尖若蹙眼波如水,眉眼好象林妹妹。水蛇腰削肩膀,这一个丫头她是谁。二爷,你若画得我晴雯呵,眉眼更象那林姑娘,岂不是碍了太太的眼,添了我晴雯的罪。}
\end{lpar}

\section{笞宝玉}

\begin{lpar}
\lver{贾 政}{陪贵客你作萎缩状,陪丫头你倒脸生光。自古道世事通明皆学问,人情练达即文章。可叹你人情世故俱不学,仕途经济撇一旁。将来不过酒色徒,于家于国都无望。}
\lver{贾 政}{在家里行为乖僻背训教,在外边无法无天又招摇。那琪官是王爷驾前承奉人,你胆敢引逗他出府逃。小奴才你不替祖宗增光彩,却祸及于我添烦恼。}
\lver{长府官}{白纸难把烈火包,公子你何苦瞒得牢。现有真凭实据在,他赠你汗巾还系你腰。}
\lver{贾宝玉}{大人既然知底细,宝玉不妨从实告。那琪官厌倦台上鬻歌舞,他厌倦台下卖欢笑。再不愿厕身优伶,离王府避居东郊。买几间竹篱茅舍,他愿作个闲乐渔樵。}
\lver{贾宝玉}{可叹你纵有行者神通广,难逃如来五指掌。}
\lver{贾 政}{天哪,想我贾府诗礼簪缨之族,富贵功名之家,竟出了这个不忠不孝的逆子,他不能光灿灿胸悬金印,他不能威赫赫爵禄高登。却与那丫头戏子结朋友,作出了诲盗诲淫丑事情。不如今日绝狗命,免将来弑父又弑君。今日打死忤逆子,明日我情愿剃度入空门。快与我活活打死休留情,免将来辱没祖宗留祸根。}
\end{lpar}

\section{闭门羹}

\begin{lpar}
\lver{薛宝钗}{饭后闲步到怡红院,嘘寒问暖把宝玉探。}
\lver{袭 人}{老爷是毒打二爷家规严,为的是望子成龙把名显。谁知他好了疮疤忘了痛,未见他本性改半点。我叹的是燕子做窠空劳碌,枉有这知寒送暖心一片。}
\lver{薛宝钗}{你是知心悉意将他待,忠肝义胆人共见。非是我当面夸你好,也常听太太说你贤。常言道急水总有回头浪,宝兄弟他总有一日性情变。青云有路终须到,飞黄腾达待来年。}
\lver{林黛玉}{宝玉被笞身负伤,荣国府多的是无情棒。他是皮肉伤愈心未愈,我是三朝两夕勤探望。}
\lver{合 唱}{一声呼叱半身凉,独立花径心凄惶。}
\lver{林黛玉}{人说是大树底下好遮荫,我却是寄人篱下气难扬。只因我无依无靠难自主,才受他薄情薄面冷如霜。我是丫头面前也受气,低头忍吃……}
\lver{合 唱}{闭门羹。}
\end{lpar}

\section{葬花}

\begin{lpar}
\lver{薛宝钗}{四月天气雨乍晴,陪着老太太来游春。宝钗理该共作伴,}
\lver{王熙凤}{龙女应当陪观音。}
\lver{薛姨妈}{人说四月春将去,我看是正当美景和良辰。}
\lver{贾 母}{老年虽有恋春意,怎奈是白发已非赏花人。}
\lver{薛宝钗}{说什么白发已非赏花人,依我看啊,老太太越活越年轻。}
\lver{王熙凤}{长生不老活下去,赛过南极老寿星。}
\lver{合 唱}{看不尽满眼春色富贵花,说不完满嘴献媚奉承话。谁知园中另有人,偷洒珠泪葬落花。}
\lver{林黛玉}{绕绿堤拂柳丝穿过花径,听何处哀怨笛风送声声。人说道大观园四季如春,我眼中却只是一座愁城。看风过处落红成阵,牡丹谢芍药怕海棠惊。杨柳带愁桃花含恨,这花朵儿与人一般受逼凌。我一寸芳心谁共鸣,七条琴弦谁知音。我只为惜惺惺怜同命,不教你陷落污泥遭蹂躏。且收拾起桃李魂,自筑香坟葬落英。花落花飞飞满天,红消香断有谁怜。一年三百六十天,风刀霜剑严相逼。明媚鲜艳能几时,一朝飘泊难寻觅。花魂鸟魂总难留,鸟自无言花自羞。愿侬此日生双翼,随花飞到天尽头。天尽头何处有香丘,未若锦囊收艳骨,一抔净土掩风流。质本洁来还洁去,不教污淖陷渠沟。侬今葬花人笑痴,他年葬侬知是谁。一朝春尽红颜老,花落人亡两不知。}
\lver{贾宝玉}{想当初妹妹从江南初来到,宝玉是终日相伴共欢笑。我把那心上的话儿对你讲,心爱的东西凭你挑。还怕那丫鬟服侍不周到,我亲自桩桩件件来照料。你若烦恼我耽忧,你若露齿我先笑。我和你同桌吃饭同床睡,象一母所生的亲同胞。实指望亲亲热热直到底,总见得我俩情谊比人好。谁知道妹妹人大你心也大,如今是你斜着眼睛把我瞧。三朝四夕不理我,使宝玉失魂落魄担烦恼。我有错你打也是骂也好,为什么远而避之将我抛。你有愁诉也是说也好,为什么背人独自你常悲嚎。你叫我不明不白鼓里蒙,我就是为你死了,也是个屈死的鬼魂冤难告。}
\lver{林黛玉}{我不顾苍苔滑天色昏,来访你秉烛共谈心。谁知道受了你丫头言欺凌,尝了你怡红院里闭门羹。撇下我满目凄凉对院门,遍体生寒立花径。那一日你蒙着耳朵不理人,今日又何必指着鼻子把誓盟。}
\end{lpar}

\section{试玉}

\begin{lpar}
\lver{紫 鹃}{看宝玉虽是有情人,是真是假还难分。他和姑娘好一阵又歹一阵,不知道他究竟安的什么心。我紫鹃今日倒要试一试,放一把火,炼他一炼,看他是黄铜还是金。}
\lver{贾宝玉}{你红嘴白牙胡乱云,妹妹是苏州原籍早无亲。老太太怜惜外孙女,千里接归伴晨昏。她离不开潇湘馆中千竿竹,怎会去姑苏城内旧墙门。}
\lver{紫 鹃}{你以为贾府族大人丁旺,难道说别人族中就无靠傍。你可知借来的东西总要还,接来的亲戚住不长。水流千里要归大海,燕子总有它旧画梁。等姑娘将来出阁时,自然要送还到故乡。姓林的不能在贾府住一世,听说是林家明春来接姑娘。}
\lver{贾宝玉}{你要去连我也带了去……除了林妹妹,凭是谁不许他姓林……哎呀船在那边等……林妹妹她从今以后去不成。}
\end{lpar}

\section{劝黛}

\begin{lpar}
\lver{紫 鹃}{你是少兄弟没父母,全仗你自定主意自张罗。春风会吹老梨花脸,光阴它轻轻在溜过。天底下王孙公子沙般多,都是那三妻四妾朝秦暮楚。我是相敬如宾的见得少,弃旧恋新看得多。真所谓万两黄金容易得,难得知心人一个。你和他一桌子吃饭一凳子坐,性情脾气摸得破。啊姑娘啊,眼前有人不张罗,你还点起灯笼找哪个。}
\lver{林黛玉}{一张小嘴开了河,絮絮叨叨把牙齿磨。把你退还给外祖母,免得你在我耳边唱山歌。}
\lver{林黛玉}{好紫鹃句句话儿含意长,她窥见我心事一桩桩。想黛玉寄人篱下少靠傍,还不知叶落归根在哪乡。老太太虽然怜惜我,总不是可恃宠撒娇象自己的娘。舅父母是宾客相待隔层肉,凤姐姐是里面尖来外面光。园中姐妹虽相好,总是那各母所生各心肠。知心人只有宝哥哥,从小就耳鬓厮磨成一双。几年来心贴心儿把日月过,情深如海难测量。因此我愿为春蚕自作茧,我为他日吐情丝夜织网。心中事牵肠挂肚推不开,好姻缘又似近身又渺茫。若说今生没奇缘,为什么合一付心肝合一付肠。若说今生有奇缘,为什么隔一座高山隔一堵墙。不由人痴痴想,我只有心坎里深把哑谜藏。}
\end{lpar}

\section{王熙凤献策}

\begin{lpar}
\lver{王夫人}{我看林姑娘啊,虽是有貌又有才,只恐怕多愁多病福分浅。那宝丫头倒是生得德容皆备有福相,品格端方十分贤。}
\lver{王熙凤}{更有金锁配宝玉,是一对天生的并蒂莲。能助家和万事兴,助得宝玉富贵全。}
\lver{袭 人}{宝二爷若娶宝姑娘,这真是天造地设配成双。二爷得了百年福,奴才也沾一线光。怎奈是金玉配恐生风浪,我知道他心里只有个林姑娘。曾记得沁芳桥畔事一桩,他曾经把我错当作林姑娘。说什么为你弄成一身病,睡里梦里不相忘。曾记得紫鹃一句玩笑话,他掀起了黄河千层浪。如今若知娶亲的事,恐天大的祸事也会闯。倒不如未曾落雨先带伞,老太太,你能提防处且提防。}
\lver{王熙凤}{这换斗移星不费难,宝玉他在病中怎识巧机关。红盖头遮住美红颜,扶新人可用紫鹃小丫环。等到那酒阑人也散,生米煮成熟米饭。总不见在销金帐中把脸翻,总不见鸳鸯枕上起波澜。}
\end{lpar}

\section{傻丫头泄密}

\begin{lpar}
\lver{林黛玉}{风萧萧兮秋气深,忧心忡忡兮独沉吟。望故乡兮何处,倚栏杆兮涕沾襟。}
\lver{合 唱}{好一似塌了青天沉了陆地,魂似风筝断线飞。眼面前桥断树倒石转路迷,难分辨南北东西。}
\end{lpar}

\section{黛玉焚稿}

\begin{lpar}
\lver{紫 鹃}{与姑娘情如手足常厮守,这模样叫我紫鹃怎不愁。端药给你推开手,水米未曾入咽喉。镜子里只见你容颜瘦,枕头边只觉你泪湿透。姑娘啊,想你眼中能有多少泪。怎禁得冬流到春夏流到秋。姑娘啊,你要多保养再莫愁,把天大的事儿放开手。保养你玉精神花模样,打开你眉上锁腹中忧。}
\lver{林黛玉}{你好心好意我全知,你曾经劝过我多少次。怎奈是一身病骨已难支,满腔愤怨非药治。只落得路远山高家难归,地老天荒人待死。}
\lver{紫 鹃}{姑娘你身子乃是宝和珍,再莫说这样的话儿痛人心。世间上总有良药可治病,更何况府中都是痛你的人。老祖宗当你掌上珍,众姐妹贴近你的心。}
\lver{林黛玉}{紫鹃你休提府中人,这府中谁是我知冷知热亲。难为你知冷知热知心待,你问饥问饱不停闲。你为我眼皮儿终夜未曾合,你把我骨肉亲人一样待。老太太派你服侍我这几年,我也将你当作我的亲妹妹。}
\lver{林黛玉}{我一生与诗书做了闺中伴,与笔墨结成骨肉亲。曾记得菊花赋诗夺魁首,海棠起社斗清新。怡红院中行新令,潇湘馆内论旧文。一生心血结成字,如今是记忆未死墨迹犹新。这诗稿不想玉堂金马登高第,只望它高山流水遇知音。如今是知音已绝诗稿怎存,把断肠文章付火焚。这诗帕原是他随身带,曾为我揩过多少旧泪痕。谁知道诗帕未变人心变,可叹我真心人换得个假心人。早知人情比纸薄,我懊悔留存诗帕到如今。万般恩情从此绝,只落得一弯冷月照诗魂。}
\lver{紫 鹃}{怪不得病榻边只有孤灯陪,却原来鹁鸪鸟都拣那旺处飞。铁心肠那管人死活,来探望的人儿却有谁。宝玉,真想不到你也是骨如寒冰心似铁,折了新桃忘旧梅。}
\lver{紫 鹃}{宝二爷娶亲瞒不了谁,又何必人未断气把命催。那边是一片喜气人如蚁,聪明能干一大堆。你要我紫鹃有何用,这锦上添花我又不会。我紫鹃今日里,只愿听这病榻旁边断肠话,决不捧那洞房宴上合欢杯。但等姑娘断了气,该把我粉身碎骨我也不皱眉。}
\lver{林黛玉}{笙箫管笛耳边绕,一声声犹如断肠刀。他那里是花烛面前相对笑,我这里是长眠孤馆谁来吊。多承你伴我月夕共花朝,几年来一同受煎熬。实指望与你并肩共欢笑,谁知道风雨无情草木凋。从今后你失群孤雁向谁靠,只怕是寒食清明梦中把我姑娘叫。我质本洁来还洁去,休将白骨埋污淖。}
\end{lpar}

\section{金玉良缘}

\begin{lpar}
\lver{贾宝玉}{林妹妹,今天是从古到今,天上人间,是第一件称心满意的事啊,我合不拢笑口将喜讯接,数遍了指头把佳期待。总算是东园桃树西园柳,今日移向一处栽。此生得娶你林妹妹,心如灯花并蕊开。往日病愁一笔勾,今后乐事无限美。从今后与你春日早起摘花戴,寒夜挑灯把谜猜。添香并立观书画,步影随月踏苍苔。从此后俏语娇音满室闻,如刀断水分不开。这真是银河虽阔总有渡,牛郎织女七夕会。}
\lver{王熙凤}{做新郎总该懂温柔,休惹得新娘气带羞。多生喜欢你少痴傻,随缘随分莫贪求。老祖宗都是为你好,你须懂得孝顺乃是第一筹。}
\lver{贾宝玉}{我爱她敬她都来不及,怎会使她气带羞。今日我十分喜减去一身病,百炼钢早化作绕指柔。}
\lver{王夫人}{愿你俩相敬如宾到白头,这才是父母之心不辜负。}
\lver{贾宝玉}{虽然这红盖头啊,遮住你面似芙蓉眉如柳,却遮不住你心底春光往外透。}
\lver{贾宝玉}{我以为百年好事今宵定,为什么月老系错了红头绳。为什么梅园错把杏花栽,为什么鹊巢竟被鸠来侵。莫不是老祖宗骗我假做亲,宝姐姐她赶走我的心上人。定气息奄奄你十分病,泪如滚水你煎着心。}
\lver{贾宝玉}{我和妹妹都有病,两个病源一条根。望求你把我们放在一间屋,也好让同病相怜心连心。活着也能日相见,死了也可葬同坟。老祖宗啊,我天下万物我无所求,只求与妹妹共死生。}
\lver{合 唱}{好一条调包计偷柱换梁,只赢得惨红烛映照洞房。}
\end{lpar}

\section{哭灵出走}

\begin{lpar}
\lver{贾宝玉}{金玉良缘将我骗,害妹妹魂归离恨天。到如今人面不知何处去,空留下素烛白帏伴灵前。林妹妹啊,如今是千呼万唤唤不归,上天入地难寻见。可叹我生不能临别话几句,死不能扶一扶七尺棺。林妹妹,想当初你是孤苦伶仃到我家来,只以为暖巢可栖孤零雁。我和你情深犹如亲兄妹,那时候两小无猜共枕眠。到后来我和妹妹都长大,共读《西厢》在花前。宝玉是剖腹掏心真情待,妹妹你心里早有你口不言。到如今无人共把《西厢》读,可怜我伤心不敢立花前。曾记得怡红院尝了闭门羹,你是日不安心夜不眠。妹妹呀,你为我是一往情深把病添,我为你是睡里梦里常想念。好容易盼到洞房花烛夜,总以为美满姻缘一线牵。想不到林妹妹变成宝姐姐,却原来你被逼死我被骗。实指望白头能偕恩和爱,谁知晓今日你黄土垅中独自眠。林妹妹啊,自从居住大观园,几年来你是心头愁结解不开。落花满地令你惊,冷雨敲窗你病未眠。你怕那人世上风刀和霜剑,到如今它果然逼你丧九泉。}
\lver{紫 鹃}{相当初姑娘病重无人理,床前只有我知心婢。她这边是冷屋鬼火三更泣,你那边是洞房春暖一天喜。只听她恨声呼宝玉,这辛酸的事儿我牢牢记。宝二爷,你来迟了,你来迟了,人死黄泉难扶起。姑娘啊……}
\lver{贾宝玉}{问紫鹃妹妹的诗稿今何在,}
\lver{紫 鹃}{如片片蝴蝶火中化。}
\lver{贾宝玉}{问紫鹃妹妹的瑶琴今何在,}
\lver{紫 鹃}{琴弦已断你休提它。}
\lver{贾宝玉}{问紫鹃妹妹的花锄今何在,}
\lver{紫 鹃}{花锄虽在谁葬花。}
\lver{贾宝玉}{问紫鹃妹妹的鹦哥今何在,}
\lver{紫 鹃}{那鹦哥叫着姑娘学着姑娘生前的话。}
\lver{贾宝玉}{那鹦哥也知情和义,}
\lver{紫 鹃}{世上的人儿不如它。}
\lver{贾宝玉}{九州生铁铸大错,一根赤绳把终身误。天缺一角有女娲,我心缺一块难再补。你已是无瑕白玉遭泥陷,我岂能一股清流随俗波。从今后你长恨孤眠在地下,我怨种愁根永不拔。人间难栽连理枝,我和你世外去结并蒂花。}
\lver{合 唱}{抛却了莫失莫忘通灵玉,挣脱了不离不弃黄金锁。离开了苍蝇竞血肮脏地,撇开了黑蚁争穴富贵窠。}
\end{lpar}

\end{document}
