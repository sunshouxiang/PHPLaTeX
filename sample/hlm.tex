\documentclass{article}

\usepackage[font=truetype,pinyin=math,href=yes,titlesec=yes]{xepkgs}
\usepackage[left=2.5cm,right=2.5cm,top=3cm,bottom=3cm,dvipdfm]{geometry}
\usepackage{indentfirst}
\usepackage{longtable}

\titleformat{\section}{\heiti\Large}{\CJKnumber{\thesection}}{1em}{}
\newenvironment{lpar}{\vspace{-1ex}\begin{longtable}{lp{0.85\columnwidth}}}{\end{longtable}}
\newcommand\lver[2]{\vspace{1ex}#1&#2\\}
\newcommand\lric[1]{{\color{red}#1}}

\iffalse%%{
function check_words($s)
{
    $level = 0;
    $nr = strlen($s);
    for ($i=0; $i<$nr; $i++) {
        if ($s[$i] == '{')
            $level++;
        elseif ($s[$i] == '}')
            $level--;
        if ($level < 0 || $level > 1)
            return false;
    }
    if ($level != 0)
        return false;
    return true;
}

function remove_comment_lines(&$lines)
{
    $clean_lines = array();
    foreach ($lines as $line) {
        if (preg_match("/^\s*%/", $line) <= 0) {
            $clean_lines[] = $line;
        }
    }
    return $clean_lines;
}

function render_scene($s)
{
    $lines = explode("\n", $s);
    $lines = remove_comment_lines($lines);
    if (count($lines) < 1)
        return;
    echo "\\section{{$lines[0]}}\n\n";
    array_shift($lines);
    echo "\\begin{lpar}\n";
    foreach ($lines as $line) {
        if (preg_match("/^(.*?):(.*)/", $line, $fields) <= 0) {
            echo "Error: $line\n";
            exit(1);
        }
        $person = $fields[1];
        $words = $fields[2];
        if (!check_words($words)) {
            echo "Error: $line\n";
            exit(1);
        }
        $words = str_replace("<", "\\lric{", $words);
        $words = str_replace(">", "}", $words);
        echo "\\lver{{$person}}{{$words}}\n";
    }
    echo "\\end{lpar}\n\n";
}
\fi%%}

\begin{document}

\begin{center}\heiti\LARGE 越剧《红楼梦》唱词\end{center}
\kaiti\large

\ 

\setlength\LTleft\parindent

%%<<<*scenes foreach ($scenes as $scene) { render_scene($scene); }
黛玉进府
紫 鹃:啊呀,林姑娘来了。啊呀你们快来看呀,林姑娘来了……
众 人:啊,老太太……林姑娘来了……
内 声:林姑娘走好,林姑娘走好……
合 唱:<乳燕离却旧时窠,孤女投奔外祖母。>
林黛玉:外祖母家确与别家不同。
合 唱:<记住了不可多说一句话,不可多走一步路。>
贾 母:啊,我的外孙女来了,在哪里呀,在哪里呀?外孙女儿!
林黛玉:外祖母。
贾 母:外孙女儿。
林黛玉:外祖母……
贾 母:我的心肝宝贝呀……<可怜你年幼失亲娘,孤苦伶仃实堪伤。又无兄弟共姐妹,似一枝寒梅独自放。今日里接来娇花倚松栽,从今后,在白头外婆怀里藏。>
王夫人:是啊。
贾 母:这是你的二舅母,快过去见过。
林黛玉:是。拜见二舅母。
王夫人:啊呀,不消了,外甥女儿,快起来吧,来,这旁坐下。
王熙凤:啊,林姑娘来了吗,真的来了啊。在哪里呀?啊呀,哈哈哈哈。我来迟了,来迟了,未曾迎接远客。啊呀,老祖宗,我来迟了。<昨日楼头喜鹊噪,今朝庭前贵客到。>
贾 母:你呀,不认识她。她是我们这里有名的一个“泼辣货”,南京人所谓“辣子”,你只要叫她一声“凤辣子”就是了。
王夫人:她就是你琏二嫂子,学名唤做王熙凤。
林黛玉:拜见二嫂子。
王熙凤:噢,起来,起来起来。啊呀,好一个妹妹。<休怪我一双凤眼痴痴瞧,似这般美丽的人儿天下少。哪像个老祖宗膝前的外孙女,分明是玉天仙离了蓬莱岛。怪不得我家的老祖宗,在人前背后常夸耀。唉,只是我妹妹好命苦,姑妈偏就去世早。>
贾 母:嗳,
王夫人:快不要伤心。
贾 母:<我一天愁云方才消,你何必又招我烦恼。>
王熙凤:哎呀真是,我一见了妹妹,一心都在她身上,又是欢喜,又是伤心,竟忘了老祖宗了。老祖宗,喏,该打,该打,该打!
众 人:哈哈哈……
王熙凤:噢,妹妹,你如今来到这里呀,<休当作粉蝶儿寄居在花丛,这家中就是你家中。你要吃要用把嘴唇动,受委屈告诉我王熙凤。>
林黛玉:多谢嫂子费心。
内 声:宝二爷回来啦,啊呀,宝二爷回来了……
贾宝玉:老祖宗安。
贾 母:起来。
贾宝玉:太太安。
王夫人:起来。
贾 母:宝玉啊,
贾宝玉:啊?
贾 母:家里来了客人,怎么不过来见过林妹妹?
王夫人:是啊,快过去见过林妹妹。
贾宝玉:林妹妹。<天上掉下个林妹妹,似一朵轻云刚出岫。>
林黛玉:<只道他腹内草莽人轻浮,却原来骨格清奇非俗流。>
贾宝玉:<闲静犹似花照水,行动好比风拂柳。>
林黛玉:<眉梢眼角藏秀气,声音笑貌露温柔。>
贾宝玉:<眼前分明外来客,心底却似旧时友。>哎,这个妹妹,我好像曾看见过的。
贾 母:啊呀,又要胡说了,你何曾见过啊?
贾宝玉:虽没见过,看见面善,心里好像认识的一般呢。
贾 母:好啊,哈,这样么以后在一起就和睦了。坐下坐下。
贾宝玉:妹妹,你读过书吗?
林黛玉:只上过一年学。
贾宝玉:噢,尊名?
林黛玉:名唤黛玉。
贾宝玉:表字呢?
林黛玉:无字。
贾宝玉:无字?好,我送妹妹一字,莫若“颦颦”两字极妙啊。
王熙凤:哎,什么叫“颦颦”啊?
贾宝玉:《古今人物通考》上说:“西方有石名黛,可做画眉之墨。”看,妹妹眉尖若蹙,取这个字岂非甚美呀?
王熙凤:只怕又是杜撰的。
贾宝玉:嗳,除了四书五经,杜撰的也太多了。妹妹,你有玉没有呀?
林黛玉:我没有玉。你那块玉也是件稀罕之物,岂能人人都有?
贾宝玉:什么稀罕的东西,人的高下不识,还说灵不灵呢。我可不要这东西。
王夫人:宝玉,你……
贾 母:宝玉,孽障!你生气,要打骂人容易,何苦去摔你那命根子哦!
贾宝玉:家里姐姐妹妹都没有,只有我有,我说没趣嘛。今天来了神仙似的妹妹,她也没有,可见这不是好东西!
王夫人:宝玉啊,快戴上。
贾 母:我不要。
王夫人:宝玉,宝玉。
贾 母:宝玉!
王熙凤:宝兄弟,戴上去,啊。
王夫人:宝玉,当心你爹知道,快戴上。
王熙凤:啊呀,宝兄弟啊,老祖宗不是常常说的吗,这宝贵家业就指望着你这个命根子呢!
贾 母:对呀!

识金锁
贾宝玉:宝姐姐请啊。
薛宝钗:宝兄弟请啊。<老太太跟前请罢安,把林家妹妹来探望。>
贾宝玉:林妹妹,林妹妹,宝姐姐来看你了呀……<这正是上庙不见土地神,>
薛宝钗:<略等片刻又何妨。>
贾宝玉:宝姐姐请坐啊。
薛宝钗:咦,宝兄弟,你颈上挂的那块玉,我虽然听见过,却未曾细细地赏鉴过,今天倒要见识一下。
贾宝玉:好啊。
薛宝钗:“莫失莫忘,仙寿恒昌”,“莫失莫忘,仙寿恒昌”……咦,你看得发呆作什么?
莺 儿:我听这两句话倒像姑娘金锁上的两句话是一对呢。
贾宝玉:宝姐姐,你金锁上面也有几个字,今天也让我赏鉴赏鉴。
薛宝钗:你不要听她讲,没有什么字的。
贾宝玉:宝姐姐,你看了我的,岂可不让人家看你的呢?
薛宝钗:我这个没有什么好看的呀。
贾宝玉:让我看看呀,看看呀。
薛宝钗:喏,只是上面有两句吉利话罢了。
贾宝玉:“不离不弃,芳龄永继”,呀,这两句话,真像和我是一对喏。好香啊,宝姐姐,你的衣服熏得好香啊。
薛宝钗:我最怕熏香,好好的衣裳为什么要熏呢?
贾宝玉:那是什么香呢?
薛宝钗:噢,想是我早晨起来,吃的“冷香丸”的香气吧。
贾宝玉:冷香丸?那么好闻,给我一丸尝尝吧。
薛宝钗:你又混闹了,药怎么可以瞎吃的呀?
贾宝玉:宝姐姐,给我一丸尝尝。
薛宝钗:你又混闹了,药怎么可以瞎吃的呢!
贾宝玉:这香气好闻呀。
薛宝钗:咦,怎么林妹妹到这般时候还没有回来,宝兄弟,我不等她了,改日再来望她吧。
贾宝玉:好吧,我送宝姐姐出去,改日再来吧。
薛宝钗:请。
贾宝玉:请吧。呀,林妹妹,方才宝姐姐来看你,你上哪里去了呀?怎么,睡着了?早饭刚刚吃好又要睡觉了。
林黛玉:你吵我做什么?
贾宝玉:好妹妹你看呀,<春光如锦不去赏,合起眼皮入睡乡,饭后贪睡易积食,>来,<我替你解闷去寻欢畅。>
林黛玉:你到别处去玩吧。
贾宝玉:我上哪里去呀?我看见他们怪腻的。
林黛玉:好,你既愿在这里,<就老老实实端正坐,休像蜜糖黏在人身上。>
贾宝玉:好,我也躺着。
林黛玉:你躺着吧。
贾宝玉:没有枕头啊,好,我们两个合一个枕头吧。
林黛玉:放屁,外面屋子里有的是枕头。
贾宝玉:我不要嘛,外面的枕头都是肮脏老婆子们用的呀。
林黛玉:你啊,真是我命中的魔星,喏。咦,你脸上又是给谁的指甲划破了?
贾宝玉:不是的,方才搽了点胭脂膏。
林黛玉:你又在做这些事了,要是传到舅舅耳朵里去,大家都不得安心了。
贾宝玉:嗯,好香啊,这香气奇怪呀,又不是香饼子的香,又不是香袋儿的香,哎,这是什么香啊?
林黛玉:难道我有什么奇香不成?好,就算我有奇香,我问你,你有暖香没有?
贾宝玉:什么暖香?
林黛玉:啊,蠢材啊蠢材,你有“玉”,人家就有“金”来配你,人家有“冷香”你就没有“暖香”去配她?
贾宝玉:好,我说了一句话,你拉上了这么许多,我今天不给你个厉害喏……
林黛玉:哦,哈……

读西厢
贾宝玉:谁?
茗 烟:哎,二爷,原来你在这里。把人家找得好苦啊。
贾宝玉:你到这里来来做什么?
茗 烟:二爷,你上次叫我弄的书,我给你拿来了。
贾宝玉:在哪里?
茗 烟:喏。
贾宝玉:《西厢记》?
茗 烟:二爷,对不对呀?
贾宝玉:对,对呀,对对对对对!
茗 烟:二爷,昨天你在老爷面前做的诗,人人都说你那些诗做的好,亏得老爷也喜欢。昨天你在老爷面前得了彩头,今天也该赏赏我啦。
贾宝玉:好,明天给你一吊钱。
茗 烟:谁没见过一吊钱啊,二爷,把这个象牙雕刻的,赏给我吧?
贾宝玉:好,拿去拿去。
茗 烟:把这只荷包也送给我吧?
贾宝玉:唉,这荷包是林妹妹送给我的呀,谁也不须拿。
茗 烟:二爷。
贾宝玉:喏,这个你拿去吧。去去去去去。
贾宝玉:这样的好书,老爷却不许我读,我今天背地里<偏要读它个爽快。读遍书斋经与史,难得《西厢》绝妙词。羡张生,琴心能使莺莺解,慕莺莺,深情更比张生痴。叹宝玉,身不由己圈在此,但愿得今晚梦游普救寺。>
林黛玉:咦!
贾宝玉:哦,是你呀。
林黛玉:你在这里做什么?噢,原来躲在这里用功,这样一来呀,可要蟾宫折桂了呢?
贾宝玉:你取笑我做什么?你不是不知道,我最讨厌诓功名、混饭吃的八股文章,你还提这些!
林黛玉:不是那些书,那又是什么书呢?不要在我面前弄鬼了,趁早给我看看。
贾宝玉:好妹妹,我在这里看这个书,除了花鸟知道,没有一个人知道。给你看我是不怕的,哎,你好歹不要告诉人。
林黛玉:什么书啊?
贾宝玉:真正是好文章!你要是看了,连饭也不想吃呢。
林黛玉:《西厢记》!
贾宝玉:嘘!轻一点。好妹妹,真正是好文章。你看,你说好不好?
林黛玉:嗯。
贾宝玉:啊,妹妹,<我是个多愁多病身啊,你就是那倾国倾城的貌!>
林黛玉:啊呀,你该死的,<胡说八道,弄出这淫词艳曲来调笑,混帐的话儿欺侮人,我可要到舅舅跟前将你告。>
贾宝玉:好妹妹,<我无非过目成诵顺口念,好妹妹你千万饶我这一遭。我若有心欺侮你,>好,明朝让我跌在这池子里,让癞头鼋<把我吞吃掉。>
林黛玉:<那张生,一封书敢于退贼寇;那莺莺,八行笺人约黄昏后;那红娘,三寸舌降服老夫人;那惠明,五千兵馅做肉馒头。我以为你也胆如斗,呸,原来是个银样蜡枪头。>
贾宝玉:好!你说说,你这个呢,我也要去告诉去。
林黛玉:你说你能过目成诵,难道我就不能一目十行么?你去呀,你去呀。
贾宝玉:好妹妹,我们不谈这些了好吗?我们坐下来谈别的吧。好妹妹,我上次到你房里来,看见你又在做针线,你做个香袋送给我好不好?
林黛玉:那要看我高兴不高兴。
贾宝玉:你送我个香袋,我也送你个好东西。这是北静王送给我的,是皇上赐下来的呀。
林黛玉:什么臭男人拿过的,我不要这东西!
贾宝玉:你不要这东西,我可要你的香袋。要,我要香袋,哎,香袋。
林黛玉:你要一个香袋,那很容易,横竖今后有人会替你做了,人家比我又会念,又会做,又会说会笑……
贾宝玉:你又来了。你这个人哪,难道连“亲不隔疏,后不僭先”都不知道?那第一件,我们两人是姑舅亲,宝姐姐是两姨亲,论亲戚也比你远;第二件,你先来,我们两人一桌吃,一床睡,从小一起长大,她是才来的呀,岂可为了她而疏远你的呢?
林黛玉:啐!我难道叫你去疏远她,那我成了什么人了呢?我为的是我的心。
贾宝玉:我也为的是我的心,难道只知道你的心,而不知道我的心不成?
林黛玉:今天天气分明冷了一些,你穿得这样单薄,回头冷了怕又要伤风。
贾宝玉:看,看你自己也穿得那么单薄。

不肖种种
贾宝玉:唉,<每曰里送往迎来把客陪,焚香叩头祭祖先。垂手恭敬听教诲,味同嚼蜡读圣贤。这饵名钓禄的臭文章,读得我头昏目眩实可厌。>
晴 雯:唉,孙悟空套上了紧箍咒,没法子噢。再读一会儿吧,我来替你打扇。
贾宝玉:唉,八股八股,把人害苦噢!晴雯……
晴 雯:嗯?
贾宝玉:晴雯,你的眉毛是谁替你画成这个样子的? 
晴 雯:我自己画的。 
贾宝玉:嗯,画的一点都不美,来,我来替你改画一下。 
晴 雯:小祖宗,你读书要紧。 
贾宝玉:你让我解解闷吧。 
晴 雯:好,就让你画吧。 二爷,你画便画噢,可造成不要把我的眉毛画得跟林姑娘一样!
贾宝玉:为什么?
晴 雯:太太不喜欢。
贾宝玉:哦?
晴 雯:有一天,太太到园中来,看见了我就虎着脸,皱着眉头,问袭人,她说:<眉尖如蹙、眼波如水,眉眼好像林妹妹,水蛇腰、削肩膀,这个丫头她是谁?>
贾宝玉:不要管它,来。
晴 雯:二爷,你若画得我晴雯呵,<眉眼更象那林姑娘,岂不是碍了太太的眼,添了我晴雯的罪。>
贾宝玉:奇怪,难道眉眼生得好看一点,也竟会得罪人了吗?哼!我偏偏要画得像林妹妹一样。来!
贾宝玉:哎,你们来看,你们来看。<你看我画得美不美?>
袭 人:<去问你家林妹妹!>
晴 雯:噢,<是谁家?香醋辣椒一起炒,我闻到,>嗯,酸辣辣的<一股辣椒味!>
袭 人:看我不撕掉你这张嘴!
晴 雯:二爷,二爷……
贾宝玉:哎,满屋子的人啊,就数她最会磨牙齿。
袭 人:都是你二爷宠着她的缘故。哎呀,二爷呀,<常言道热心人总爱多张嘴,休怪我要把二爷劝一回。你怎可凤凰混在乌鸦队,主子替奴才去画眉。你放下正经书不念,被老爷知道定责备。>哎呀二爷,就是退一万步说,<纵然你不是真心爱读书,你也应该装出个读书样子来。>
贾宝玉:读书读书,又是读书! 
袭 人:看,天时热了,来,脱下件衣服吧?咦!这条汗巾是从哪里来啊?我怎么从来都没有看见过?说呀,二爷,是哪里来的呀?
贾宝玉:是一个朋友送给我的呀。
袭 人:朋友?是什么样的朋友竟送你这样的东西。
贾宝玉:哎呀,你少问一些好不好?
袭 人:啊呀二爷,你又不知结交上了什么三教九流的人物了,哎……
晴 雯:哎呀,好鲜艳的一条汗巾。嗯,不知我家二爷呀又在做了些什么瞒着人的事了?若有风风雨雨被上头知道,红萝卜又要算在蜡烛账上,叫我们做奴婢的晦气。
贾宝玉:你怕什么?我干的都正经事,又没去为非作歹哟。
晴 雯:那你哪里来的?
贾宝玉:不瞒你说,是忠顺王府里有个唱戏的戏子,名叫琪官,他送给我留作纪念的呀。
晴 雯:唱戏的戏子……
贾宝玉:<那琪官自幼父母双亡故,十一岁卖进戏班学歌舞,十三岁他一入侯门深似海,进了忠顺亲王府。到如今啊他名满天下是艺超群,谁能知他台上卖笑台下苦!他是侍曲陪酒心不甘,心不甘伶人当做王爷奴。我有缘相识豪侠友,又蒙他赠我汗巾“茜香罗”>
袭 人:哎呀,二爷,你怎么竟和戏子结交了朋友,做这种事情?哎呀,二爷。<平日里你不分上下贵贱,与下人共奴婢平起平坐。今日里又与戏子结朋友,岂不防品行名声被玷污!若被老爷来知晓,家法如何饶得过。>
晴 雯:呦!与戏子交个朋友,这难道这也犯了什么大罪了吗?难道说人家唱戏的一定比人家低一头,贱一些吗?<皇帝也有草鞋亲,与戏子往来有什么错?>
薛宝钗:宝兄弟……
袭 人:是宝姑娘来了。
薛宝钗:嗯,怎么啦,不欢迎客人吗?那我也不该来了。
贾宝玉:不,宝姐姐,来来来来。
袭 人:宝姑娘请坐。
贾宝玉:宝姐姐,请坐啊。
薛宝钗:你们刚才讲得这样热闹,在讲些什么?
袭 人:宝姑娘,你看,他竟和戏子结交朋友。
薛宝钗:戏子?
袭 人:是啊,那还得了吗?
薛宝钗:宝兄弟,要是你真和戏子结交,这倒叫人寒心呢。
贾宝玉:宝姐姐,你不要听这些话。哎,上次诗社你不是叫我做的诗吗?我已经作好了,请你看看。
焙 茗:二爷!
贾宝玉:什么事?
焙 茗:方才老爷叫我来传话,说贾雨村老爷明日一早要来拜访,老爷叫你准备准备,明天好会客。
贾宝玉:又是会客!
焙 茗:这是老爷吩咐的嘛!
贾宝玉:宝姐姐,你替我想想,老爷每逢接待宾客,总是要我也陪着,你说说这是为什么?
薛宝钗:当然你能迎宾接客,所以才叫你呢。
贾宝玉:我无非是个俗中又俗的俗人罢了,我真不愿意与这些“禄蠹”们来往呢!
袭 人:宝姑娘,你听听,他呀,就是这个改不了,世界上哪有一个不愿和做官人来往,倒去愿意和戏子结交朋友的道理?
薛宝钗:嗯,这倒真要改一改才好。宝兄弟,<常言道主雅客来勤,谁不想高朋能盈门。如今你尚未入仕林,你也该会会做官人。谈讲些仕途经济好学问,学会些处世做人真本领。正应该百尺竿头求上进,怎能够不务正业薄功名。>
贾宝玉:宝姐姐,老太太要玩骨牌,正没人呢!你去玩骨牌去吧!
薛宝钗:我难道是专陪人家玩骨牌的吗?
袭 人:宝姑娘,你不要理他这些。喏,上回史大姑娘也劝过他一回,他也不顾人家脸上过不去,咳了一声,提起脚来就走了。人家劝他上进,他总是骂人家什么“禄蠹”啊,你想,怎么怨得老爷不生气呢?
薛宝钗:我走了……
袭 人:宝姑娘,再坐一会吧?
薛宝钗:我去看看姨娘去。
袭 人:宝姑娘走好,宝姑娘走好,啊,宝姑娘走好啊。哎,宝姑娘真是心地宽大有涵养,幸而是宝姑娘,要是换了林姑娘啊,又不知闹得怎么样,哭得怎么样呢。提起这些,宝姑娘真是叫人敬重,可是你呀,倒与人家生分了。
贾宝玉:林姑娘从来没有说这些混帐的话。
袭 人:这难道是混帐的话吗?
贾宝玉:唉,真想不到,琼楼闺阁之中,也会染上了这种风气。
袭 人:二爷。
林黛玉:<万两黄金容易得,人间知己最难求。>
合 唱:<背地偷闻知心话,但愿知心到白头。>

笞宝玉
贾 政:好端端的,垂头丧气做什么呀?方才贾雨村来了,要见你,等你半天才出来,既出来了,又无慷慨潇洒的谈吐,显得委委琐琐的,哼。<陪贵客你做委琐状,陪丫鬟你倒脸生光。自古道世事洞明皆学问,人情练达即文章。可叹你,人情世故俱不学,仕途经济撇一旁。将来不过酒色徒,于家于国都无望。>近来是谁跟着你上学?
焙 茗:是小的焙茗。
贾 政:他读了些什么书?一定是读了些流言混话在肚子里,学了些精致的淘气!哼,等我空了,先剥了你的皮。
焙 茗:是,是。
贾 政:再和这不上进的算账!还在这里做什么,还不与我下去。
贾宝玉:是。
仆 人:禀老爷,亲王府里有人来,要见老爷。
贾 政:吩咐快请。
仆 人:是。
贾 政:不知大人驾到,未出远迎,望请恕罪。
长府官:岂敢。
贾 政:大人请。
长府官:请。
贾 政:大人请坐。
长府官:请坐。
贾 政:请问大人……
长府官:下官奉王命而来,有一事相烦老先生做主。
贾 政:望大人宣明,学生好遵谕承办。
长府官:呵呵,也不必承办,只用老先生一句话就完了。
贾 政:哪里,哪里。
长府官:我们府上有个唱戏的戏子,名叫琪官,乃是我王爷心爱的,如今三五曰不见回去,四处寻找无着,今城内众人传说,琪官与令郎宝玉相交甚厚。
贾 政:哦?
长府官:听说逃出府去,也是令郎的主意。故此相烦老先生转致令郎,请将琪官放回,一则可慰我王爷奉恳之意,那二么,嘿嘿,也免下官求觅之苦。
贾 政:请大人稍待,来人啊,唤宝玉!
贾宝玉:老爷。
贾 政:过来,该死的奴才!<在家里你行为乖僻背训教,在外边无法无天又招摇!那琪官是王爷驾前承奉人,你胆敢引逗他出府逃!小奴才,你不替祖宗增光彩,却祸及于我添烦恼!>
贾宝玉:什么琪官?我实在不晓得此事呀。
贾 政:啊……
长府官:嘿嘿……<白纸难把烈火包,公子你何苦瞒得牢。>
贾宝玉:恐为讹传,亦未见得呀。
贾 政:哎,
长府官:讹传?<现有真凭实据载,他赠你的汗巾还系你腰。>
贾 政:啊!你讲、你讲!
贾宝玉:<大人既然知底细,宝玉不妨从实告。那琪官厌倦台上鬻歌舞,他厌倦台下卖欢笑。再不愿厕身优伶,离王府避居东郊。买几间竹篱茅舍,他愿作个闲乐渔樵。>
贾 政:啊!
长府官:听你说来,他定在那里了?
贾宝玉:是。
长府官:好!我去找他,找着了便罢,若没有么再来请教,告辞!
贾 政:送大人。不许走开,回头有话问你!请大人慢走,请大人慢走。
贾宝玉:啊呀琪官,琪官那!<可叹你纵有行者神通广,难道如来五指掌。>
贾 政:站住!小奴才,来人啊,来人啊!
仆 人:老爷。
贾 政:绑起来,绑起来!
仆 人:是。
贾 政:与我取大板来,与我拿绳子来!把门都关上,若是有人传信到里面去,立刻打死!
仆 人:是。
贾 政:与我拖下去打,拖下去!天哪,天哪,想我贾府诗礼簪缨之族、宝贵功名之家,竟出了这个不中不孝的<逆子!他不能光灿灿胸悬金印,他不能威赫赫爵禄高登。却与那丫头戏子结朋友,作出了诲盗诲淫丑事情。不如今曰绝狗命,免将来弑父又弑君。今曰打死忤逆子,明曰我情愿剃度入空门。快与我活活打死休留情,免将来辱没祖宗留祸根。>拖下来!
贾宝玉:老祖宗,老祖宗快来,老祖宗快来呀!
贾 政:我今天非打死你不可!
王夫人:宝玉,宝玉!老爷,老爷!老爷,宝玉虽然该打,你也该保重,打死宝玉事小,倘若把老太太气坏了,岂非事大了?宝玉……
贾 政:夫人休提此言!我养了这个逆子,我已不孝,趁此结果了他,以免后患。
王夫人:老爷,你也该看在夫妻份上,我年已半百,只有这一孽障。我们娘儿俩不如一起死了,在阴司里也可以得个依靠!宝玉,我苦命的儿啊!
贾 政:都是你,都是你,都是你把他宠成这个样子。我今天非勒死他不可,与我拿绳子来!
王夫人:老爷,老爷,倘若我的贾珠儿还活在世上的话,你不要说打死一个宝玉,你就是打死一百个宝玉,我也不管了。如今只有这一个儿子,老爷你,你……你就饶了他吧。
贾 政:你与我放手来。
王夫人:你就饶了他吧……
贾 政:放手来!
王夫人:啊,好啊,你一定要把他勒死,好吧,你不如把我先勒死了,你把我先勒死了。
内 声:老太太到。
贾 母:你先打死我,你先打死我!宝玉!
贾 政:老太太,有什么吩咐,何必自己走出来,唤儿子进去吩咐也就是了。
贾 母:你原来和我讲话,我倒有话吩咐,只是我一生没养个好儿子,却叫我和谁说去?
贾 政:老太太,作儿子的管教他,也为了是荣宗耀祖,老太太说此话,做儿子的如何担当得起?
贾 母:呸!我只讲一句话,你就经受不起,你那样的棍子,难道宝玉就经受得起了?!
贾宝玉:老祖宗……
贾 母:宝玉啊,你这不学好、不争气的孙子哦!
王熙凤:宝兄弟。

闭门羹
薛宝钗:<饭后闲步到怡红院,嘘寒问暖把宝玉探。>
晴 雯:宝姑娘。
薛宝钗:袭人,宝兄弟在吗?
晴 雯:二爷他已经睡着了呀。
薛宝钗:睡了?
晴 雯:嗯。
袭 人:是宝姑娘来了……
薛宝钗:袭人。你在做什么?哎呀,他这么大了,还要戴这个兜肚吗?
袭 人:他的伤刚好,夜里睡觉翻来覆去的,哄他戴上这兜肚,夜里落了被也不会冻着他。
薛宝钗:真亏你想得周到。
袭 人:唉……
薛宝钗:袭人,你好端端的怎么叹起气来了?
袭 人:宝姑娘你哪里知道,<老爷是毒打二爷家规严,为的是望子成龙把名显。谁知他好了疮疤忘了痛,未见他本性改半点。我叹的是燕子做窠空劳碌,枉有这知寒送暖心一片。>
薛宝钗:你何必这样灰心呢,<你是知心悉意将他待,忠肝义胆人共见。非是我当面夸你好,也常听太太说你贤。常言道急水总有回头浪,宝兄弟他总有一日性情变。青云有路终须到,飞黄腾达待来年。>
袭 人:宝姑娘说的是,你也常为二爷费了许多心思。好姑娘,你再帮我们劝劝他吧。
薛宝钗:只要用得着我,绝不推辞。
袭 人:哎呀,宝姑娘来了半天,我连一杯茶都没有倒。好姑娘,你坐一会,我给你倒一杯茶来。
薛宝钗:不用了。
贾宝玉:什么话啊?和尚道士的话如何信的?什么金玉良缘?我偏要说是木石姻缘。宝姐姐,是你啊!
薛宝钗:宝兄弟,你身体大愈了?
贾宝玉:多谢你牵记,我全都好了。你来了好多时候了吧?
薛宝钗:我刚坐一会,我刚才还听见你在梦中骂人呢。
贾宝玉:哦?
薛宝钗:嗯,我真想不到,到这里来是来听你骂人的。
贾宝玉:是真的吗?呵呵,我自己一点都不知道呢。
薛宝钗:算了,梦中之言不足为信。宝兄弟,我一来时望望你的,那二,听说你近来作了几首新诗,我特来赏鉴一番的呀!
贾宝玉:我的诗啊任凭做的怎么好,总不及你和林妹妹呀,我正要请你品评一下呢。
袭 人:宝姑娘,那就请到里面来吧。
贾宝玉:好,请吧。
薛宝钗:请。
晴 雯:什么宝姑娘贝姑娘的,有事无事来了就坐着,叫人家半夜三更的不得好睡!
林黛玉:<宝玉被笞身负伤,荣国府多的是无情棒。他是皮肉伤愈心未愈,我是三朝两夕勤探望。>
晴 雯:谁啊,都睡着了,有事明天再来吧。
林黛玉:是我啊,还不开门。
晴 雯:凭你是谁,二爷吩咐,一概不准放进人来!
合 唱:<一声呼叱半身凉,独立花径心凄惶。>
林黛玉:<人说是大树底下好遮荫,我却是寄人篱下气难扬。只因我无依无靠难自主,才受他薄情薄面冷如霜。我是丫头面前也受气,低头忍吃……>
合 唱:<闭门羹。>

葬花
王熙凤:老祖宗走好啊!老祖宗……
薛宝钗:老祖宗走好。<四月天气雨乍晴,陪着老太太来游春啊。>
薛姨妈:<人说四月春将去,我看是正当美景和良辰。>
薛宝钗:老太太,你累了,到那边去坐一会儿吧?
贾 母:好好。<老年虽有恋春意,怎奈是白发已非赏花人。>
薛宝钗:<说什么白发已非赏花人,>依我看哪,<老太太越活越年轻。>
众 人:说的是啊!
王熙凤:<长生不老活下去,赛过南极老寿星。>
贾 母:哈哈,就凭你这张巧嘴啊,
王熙凤:啊呀老祖宗。
薛宝钗:这几年啊,我留心的看起来,二嫂自凭她怎样巧,总巧不过老太太。
众 人:是啊是啊。
贾 母:我的儿啊,如今我老了,还巧什么呢?当年我像凤丫头这般年纪倒是比她还强呢!
众 人:是啊是啊。
贾 母:姨太太,不是我当着姨太太的面奉承,千真万真,从我们家四个女孩儿算起,要说巧要说好啊,(笑)诺,都不及这宝丫头哦。
众 人:是啊是啊。
薛姨妈:老太太说偏了。
王熙凤:姨妈,这倒是真的哦,我时常听到老祖宗在背后说宝姑娘好呢?
贾 母:凤丫头啊,等一会儿你去准备一些好吃的东西,娘儿们今天所幸就乐一乐(笑)姨太太,你想吃什么,尽管告诉我,我有本事叫凤丫头办了来吃。
薛姨妈:老太太总是给她出难题,时常叫她弄来些好吃的东西来孝敬我们。
王熙凤:姑妈你休说了。我们老祖宗只是嫌人肉酸,要不嫌人肉酸,她早就把我也吃了呢!
合 唱:<看不尽满眼春色富贵花,说不完满嘴献媚奉承话。>
薛宝钗:老祖宗快来啊!
王熙凤:老祖宗走好啊!
贾 母:凤辣子啊
合 唱:<谁知园中另有人,偷洒珠泪葬落花。>
林黛玉:<绕绿堤拂柳丝穿过花径,听何处哀怨笛风送声声。人说道大观园四季如春,我眼中却只是一座愁城。看风过处落红成阵,牡丹谢、芍药怕、海棠惊。杨柳带愁、桃花含恨,这花朵儿与人一般受逼凌。我一寸芳心谁共鸣,七条琴弦谁知音。我只为惜惺惺、怜同命,不教你陷落污泥遭蹂躏。且收拾起桃李魂,自筑香坟葬落英。花落花飞飞满天,红消香断有谁怜。一年三百六十天,风刀霜剑严相逼。明媚鲜艳能几时,一朝飘泊难寻觅。花魂鸟魂总难留,鸟自无言花自羞。愿侬此日生双翼,随花飞到天尽头。天尽头何处有香丘,未若锦囊收艳骨,一抔净土掩风流。质本洁来还洁去,不教污淖陷渠沟。侬今葬花人笑痴,他年葬侬知是谁。一朝春尽红颜老,花落人亡两不知。>人说我痴,难道还有个痴的不成吗?我打量是谁?原来是这个狠心短命……
贾宝玉:妹妹慢走,我知道你不理我,看见我就避开。今天我只要和你说一句话,从今之后就撂开手吧?
林黛玉:说吧。
贾宝玉:说两句,你听不听?唉,既有今曰,何必当初!
林黛玉:当初怎么样,今曰又怎么样?
贾宝玉:<想当初妹妹从江南初来到,宝玉是终曰相伴共欢笑。我把那心上的话儿对你讲,心爱的东西凭你挑。还怕那丫鬟服侍不周到,我亲自桩桩件件来照料。你若烦恼我耽忧,你若露齿我先笑。我和你同桌吃饭同床睡,像一母所生的亲同胞。实指望亲亲热热直到底,总见得我俩情谊比人好。谁知道妹妹人大你心也大,如今是你斜着眼睛把我瞧。三朝四夕不理我,使宝玉失魂落魄担烦恼。我有错你打也是骂也好,为什么远而避之将我抛。你有愁诉也是说也好,为什么背人独自你常悲嚎。你叫我不明不白鼓里蒙啊,>我就是为你死了,<也是个屈死的鬼魂冤难告。>
贾宝玉:怎么啦,你在哭啦?
林黛玉:谁在哭。
贾宝玉:看,珠泪还滚着呢。
林黛玉:你要死啦 ,动手动脚的。
贾宝玉:说话忘了情,不觉动了手,也就顾不得死活了。
林黛玉:你既这么说,我来问你,那天我到怡红院去,你为什么不叫丫鬟开门呢?
贾宝玉:啊,此话从哪里说起呀?噢,怪不得你不理我了。好,我若敢这样对待妹妹,叫我立刻就死好了!
林黛玉:啐!那一天哪,<我不顾苍苔滑天色昏,来访你秉烛共谈心。谁知道受了你丫头言欺凌,尝了你怡红院里闭门羹。撇下我满目凄凉对院门,遍体生寒立花径。那一曰你蒙着耳朵不理人,今曰又何必指着鼻子把誓盟。>
贾宝玉:好妹妹,我实在不知道你来过呀!那天只有宝姐姐来坐过一会。定是丫头们干出来的好事!等我回去问出是谁,我定要教训教训她们哪!
林黛玉:嗯,是要教训教训才好,得罪了我倒是小事,要是以后宝姑娘来了,什么贝姑娘来了,也把她们也得罪了,事情可大了!
贾宝玉:啊呀,你还说这些话。哎,你到底是气我呢,还是咒我啊!
林黛玉:这有什么要紧呢?啊呀,筋都暴起来,还急得一脸汗呢。
贾宝玉:你放心。
林黛玉:啊,我有什么不放心的?我真不明白你的意思,你倒说说看,什么放心不放心的?
贾宝玉:啊,你果然不明白这话吗?难道我平日在你身上用的心都用错了吗?连你的意思若体贴不着,那就难怪你天天为我生气了。
林黛玉:我真不明白。
贾宝玉:好妹妹,你不要骗我。你要真地不明白这话,不但我平日白费了心,而且连你对待我的心都辜负了。唉,你总是因为不放心的缘故,才多了心,才弄了这一身的病。好妹妹,你若能宽慰些,病就会好呢。妹妹慢走,你再让我讲一句话再走好不好?
林黛玉:还有什么可以说的,你的话,我都明白了。
贾宝玉:都明白了?好妹妹,我这个心从来也不敢说,今天大胆地说出来,我就是死了也是情愿的。我为你也弄了一身的病,又不敢告诉人,只好挨着……只怕等你的病好了,我的病才会好呢。我睡里梦里也忘不了你。啊,好妹妹……啊?袭人,是你!
袭 人:这是哪里的话,你怎么啦?啊呀,二爷,看你神色不对,快回去歇息吧。

试玉
紫 鹃:<看宝玉虽是有情人,是真是假还难分。他和姑娘好一阵又歹一阵,不知道他究竟安的什么心。我紫鹃今日倒要试一试,>放一把火,炼他一炼,<看他是黄铜还是金。>
贾宝玉:紫鹃,你在想什么呀?
紫 鹃:我在想,若每日一两燕窝,在这里吃惯了,明年家里去,哪里有钱吃得起这个?
贾宝玉:谁要回去?
紫 鹃:林妹妹回苏州自己家里去咯。
贾宝玉:啊呀,你呀你呀,说什么谎来?<你红嘴白牙胡乱云,妹妹是苏州原籍早无亲。老太太怜惜外孙女,千里接归伴晨昏。她离不开潇湘馆中千竿竹,怎会去姑苏城内旧墙门。>
紫 鹃:你也太小看人了!<你以为贾府族大人丁旺,难道说别人族中就无靠傍。你可知借来的东西总要还,接来的亲戚住不长。水流千里要归大海,燕子总有它旧画梁。等姑娘将来出阁时,自然要送还到故乡。姓林的不能在贾府住一世,听说是林家明春来接姑娘。>二爷,二爷,宝二爷!啊呀,二爷……啊呀二爷不好了……
袭 人:紫鹃你闯下祸了?
紫 鹃:我只不过和他讲了一句话,他就变成这样。
袭 人:二爷,二爷!
王熙凤:宝兄弟!
贾 母:宝玉!
王夫人:宝玉,宝玉!
贾 母:袭人,你是怎样伺候的,把宝玉弄成这个样子?
袭 人:不知紫鹃“姑奶奶”说了一些什么话,二爷眼也直了,手脚也冷了,话也不会说了。
王夫人:这可不中用了啊。
贾 母:你这个死丫头,你说了些什么?
王熙凤:死丫头,你和他讲了些什么?
紫 鹃:我,我并未不敢和他讲过什么话,只是和他开个玩笑。
贾 母:真是!
紫 鹃:宝二爷,你可不能当真阿!
贾宝玉:啊!
贾 母:喔唷,还好啊!
王熙凤:死丫头,得罪了二爷,还不过去赔罪!
贾宝玉:紫鹃你不能走,你不能走!<你要去连我也带了去。>
袭 人:二爷,二爷。
贾 母:啊呀,这是怎么一回事啊?
紫 鹃:我只是和他说过一句玩笑,说是林姑娘要回苏州自己家里去。噢,二爷他就……
贾 母:啊呀呀!
内 声:老太太,林家妈妈要来看宝二爷。
贾 母:你叫林妈妈在客堂去坐一会儿。
内 声:是。
贾宝玉:啊呀,不得了,不得了!林家的人来接林妹妹来了,来接林妹妹来了!快打出去!
贾 母:宝玉啊!啊,打出去,打出去!
王熙凤:你们听见了没有?老太太吩咐,打出去,快打出去。
贾 母:啊呀,这不是林家的人。
贾宝玉:除了林妹妹,<凭是谁,不许他姓林。>
贾 母:对呀,你们听着,以后你们不准让姓林的进来,不准提到林字,听见了吗?
王熙凤:你们听见了没有,以后不准再提到林字,听见吗!
贾宝玉:啊呀,那边一只船,不是来接林妹妹的呀?<船在那边等!>
王夫人:宝玉,不是的。
贾 母:来人,叫船摇走,叫船摇走!哎呀,替我摇走啊。宝玉,那你总该放心了吧。
王熙凤:是啊,宝兄弟……
贾宝玉:紫鹃,<林妹妹她从今以后去不啊成!>
众 人:啊……
紫 鹃:二爷。

劝黛
林黛玉:紫鹃,
紫 鹃:哎。
林黛玉:你去看宝二爷的病,他好一点了没有?
紫 鹃:好倒是好一点了,不过他还记得我上次要骗他回去的事,他拦住我的手说:“活着我们一起活着,不活着就一起化灰化烟,可就不要分开了。”宝玉的心倒是真直,听见我们要回去,他就急成那个样子。噢,姑娘,我差一点忘记了,我在刚才去看二爷时,二爷叫我把一块手帕送给你,我一看是一条半新半旧他平时自己用的手帕,我说这是做什么?他说你姑娘自然会知道。这一说我也就明白了,姑娘,你也该趁早拿定个主意呀!
林黛玉:你说什么啊?
紫 鹃:我跟你也多少年了,你待我又是那样好,难道你的心事我还会不知道吗?姑娘,<你是少兄弟没父母,全仗你自定主意自张罗。春风会吹老梨花脸,光阴它轻轻在溜过。天底下王孙公子沙般多,都是那三妻四妾朝秦暮楚。我是相敬如宾的见得少,弃旧恋新看得多。真所谓万两黄金容易得,难得知心人一个。你和他一桌子吃饭一凳子坐,性情脾气摸得破。啊姑娘啊,眼前有人不张罗,你还点起灯笼找哪个。>
林黛玉:丫头,你疯了?<一张小嘴开了河,絮絮叨叨把牙齿磨。>你若这样我就不敢要你了,<把你退还给外祖母,免得你在我耳边唱山歌。>
紫 鹃:姑娘,我说的都是好话,又没叫你去为非作歹,何苦去回禀老太太,叫我吃了亏,又有什么好处呢?
林黛玉:好妹妹,你累了一天,也该去歇息了。
紫 鹃:我这就去。
林黛玉:<好紫鹃句句话儿含意长,她窥见我心事一桩桩。想黛玉寄人篱下少靠傍,还不知叶落归根在哪乡。老太太虽然怜惜我,总不是可恃宠撒娇像自己的娘。舅父母是宾客相待隔层肉,凤姐姐是里面尖来外面光。园中姐妹虽相好,总是那各母所生各心肠。知心人只有宝哥哥,从小就耳鬓厮磨成一双。几年来心贴心儿把日月过,情深如海难测量。因此我愿为春蚕自作茧,我为他日吐情丝夜织网。心中事牵肠挂肚推不开,好姻缘又似近身又渺茫。若说今生没奇缘,为什么合一个心肝合一付肠?若说今生有奇缘,为什么隔一座高山隔一堵墙?不由人痴痴想……,我只有心坎里深把哑谜藏。>

王熙凤献策
贾 母:既然你们都说宝丫头好,我的心里也愿意哦。
王夫人:老太太,我们心里虽说好,但林姑娘也要给她说了人家才好。倘若那姑娘真与宝玉有些儿私心,她知道宝玉定下宝钗,倒要防生面什么事来呢。
贾 母:自然先给宝玉娶亲,然后再给林丫头找婆家,再没有先是外人,后是自己呀。至于宝玉定亲的事么,就不许叫她知道罢了。
傻丫头:嘻……
王熙凤:你听见了,可不准传出去,若走漏了一个字,当心打断你两条腿!
贾 母:嗯。
袭 人:老太太,太太……
王夫人:袭人,好端端的,有什么委屈了,快起来说吧。
袭 人:是,这话,奴才本是不敢讲的,如今没法子只好讲了。<宝二爷若娶宝姑娘,这真是天造地设配成双。二爷得了百年福,奴才也沾一线光。怎奈是金玉配恐生风浪,我知道他心里只有个林姑娘。曾记得沁芳桥畔事一桩,他曾经把我错当作林姑娘。说什么为你弄成一身病,睡里梦里不相忘。曾记得紫鹃一句玩笑话,他掀起了黄河千层浪。如今若知娶亲的事,恐天大的祸事也会闯。倒不如未曾落雨先带伞,老太太,你能提防处且提防。>
王夫人:老太太……
贾 母:啊呀,若宝玉真是这样,这倒教人难了啊!
王熙凤:难倒不难,我想到了一个主意,不知姑妈肯不肯?
王夫人:你只管说来。
王熙凤:依我看,这件事只有个“掉包”的法子,
%%=$\overset{\mbox{贾 母}}{\mbox{王夫人}}$:\raisebox{1ex}{掉包?}
王熙凤:是呀,这个掉包的法子,只是外头一概不准提起。太太……
王夫人:这……这样能行吗?
王熙凤:只有这样。
王夫人:也罢了。
贾 母:嗳,娘儿们搞什么鬼,也该让我听听么。
王熙凤:噢,老祖宗,喏……
贾 母:只是能瞒得过么?
王熙凤:老祖宗,<这换斗移星不费难,宝玉他在病中怎识巧机关。红盖头遮住美红颜,扶新人可用紫鹃小丫环。等到那酒阑人也散,生米煮成熟米饭。总不见在销金帐中把脸翻,总不见鸳鸯枕上起波澜。>
贾 母:哈哈哈……
%王夫人:我看林姑娘啊,虽是有貌又有才,只恐怕多愁多病福分浅。那宝丫头倒是生得德容皆备有福相,品格端方十分贤。
%王熙凤:更有金锁配宝玉,是一对天生的并蒂莲。能助家和万事兴,助得宝玉富贵全。

傻丫头泄密
林黛玉:紫鹃,
紫 鹃:哎。
林黛玉:我一时出来,忘了带块手帕,你给我回去取一下,我在这里等着你。
紫 鹃:好,我就去拿来。
林黛玉:<风萧萧兮秋气深,忧心忡忡兮独沉吟。望故乡兮何处?倚栏杆兮涕沾襟……>
傻丫头:呜呜呜……
林黛玉:你好好的为什么在此啼哭,受了什么人的气了?
傻丫头:林姑娘,你评评这个理,他们说话我又不知道,我就是说错了一句话,我姐姐也就不该打我!
林黛玉:你姐姐是哪一个?
傻丫头:就是珍珠姐姐。
林黛玉:噢,你叫什么?
傻丫头:我叫傻大姐。
林黛玉:你姐姐为什么要打你,你说错什么话了?
傻丫头:为什么?还不是为了宝二爷娶宝姑娘的事情。
林黛玉:你说什么啊?
傻丫头:喏,就是为宝二爷娶宝姑娘的事情。
合 唱:<好一似塌了青天、沉了陆地,魂似风筝断线飞。眼面前桥断、树倒、石转、路迷,难分辨南北东西。>
紫 鹃:姑娘,你究竟要往哪里去啊?
林黛玉:我……我问问宝玉去!
紫 鹃:姑娘,姑娘……啊,姑娘!

黛玉焚稿
紫 鹃:姑娘,起来吃药吧。姑娘,你就吃一点吧。
林黛玉:紫鹃,你哭什么,我哪里能够死呢!
紫 鹃:姑娘,<与姑娘情如手足常厮守,这模样叫我紫鹃怎不愁?端药给你推开手,水米未曾入咽喉。镜子里只见你容颜瘦,枕头边只觉你泪湿透。姑娘啊,想你眼中能有多少泪啊,怎禁得冬流到春,夏流到秋?姑娘啊,你要多保养,再莫愁,把天大的事儿放开手。保养你玉精神,花模样,打开你眉上锁,腹中忧。>
林黛玉:<你好心好意我全知,你曾经劝过我多少次。怎奈是一身病骨已难支,满腔愤怨非药治。只落得路远山高家难归,地老天荒人待死。>
紫 鹃:不,姑娘,<姑娘你身子乃是宝和珍,再莫说这样的话儿痛人心。世界上总有良药可治病,更何况府中都是疼你的人。老祖宗当你掌上珍,众姐妹贴近你的心……>
林黛玉:不用说了。<紫鹃你休提府中人,这府中谁是我知冷知热亲!>
紫 鹃:姑娘。
林黛玉:妹妹,只有你是我最知心的了。<难为你,知冷知热知心待,你问饥问饱不停闲。你为我眼皮儿终夜未曾合,你把我骨肉亲人一样待。老太太派你服侍我这几年,我也将你当作我的亲妹妹。>
紫 鹃:姑娘,姑娘。
林黛玉:妹妹,把我的诗本子拿来。
紫 鹃:姑娘,等身体好了再看吧
林黛玉:不,拿来。
紫 鹃:姑娘,姑娘……何苦自己又生气呢。
林黛玉:火盆拿来……
紫 鹃:姑娘,多盖上一件吧,这火盆里有炭气,只怕受不住……
林黛玉:拿来。<我一生与诗书做了闺中伴,与笔墨结成骨肉亲。曾记得,菊花赋诗夺魁首,海棠起社斗清新。怡红院中行新令,潇湘馆内论旧文。一生心血结成字,如今是记忆未死墨迹犹新。这诗稿,不想玉堂金马登高第,只望它,高山流水遇知音。如今是知音已绝,诗稿怎存?把断肠文章付火焚。这诗帕原是他随身带,曾为我揩过多少旧泪痕。谁知道诗帕未变人心变,可叹我真心人换得个假心人。早知人情比纸薄,我懊悔留存诗帕到如今。万般恩情从此绝,只落得一弯冷月照诗魂。>
紫 鹃:姑娘,姑娘,……你快躺下吧。雪雁,你告诉老太太,太太,林姑娘病重,她们怎样说?
雪 雁:姐姐,你不要问了。
紫 鹃:快讲啊。
雪 雁:宝二爷真地娶……
紫 鹃:你说什么?
雪 雁:就在今夜成亲,新房都是另外收拾的。上头吩咐了,不叫我们知道。
紫 鹃:这些人怎么这样狠毒冷淡哦!<怪不得病榻边只有孤灯陪,却原来鹁鸪鸟都拣那旺处飞。铁心肠那管人死活,来探望的人儿却有谁?!>宝玉,我看她明朝死了,你拿什么脸来见我!真想不到你也是<骨如寒冰心似铁,摘了新桃忘旧梅!>
雪 雁:周妈妈来了。
周妈妈:呀,林姑娘怎么样啦?唉……
雪 雁:周妈妈,你有什么事吗?
周妈妈:我要和紫鹃姑娘说几句话。紫鹃姑娘,来来。紫鹃姑娘,方才老太太与二奶奶商量过,那边要你使唤呢。
紫 鹃:周妈妈,你先请吧,等人死了,我们自然是出来听候使唤的,只是林姑娘还有口气呢。
周妈妈:紫鹃姑娘,你这话对我说倒是使得,我可怎么去回禀上头呢?
紫 鹃:周妈妈,你只管去回禀上头说:<宝二爷娶亲瞒不了谁,又何必人未断气把命催?那边是一片喜气人如蚁,聪明能干一大堆。你要我紫鹃有何用,这锦上添花我又不会。>
周妈妈:紫鹃……
紫 鹃:我紫鹃今日里,<只愿听这病榻旁边断肠话,决不捧那洞房宴上合欢杯。但等姑娘断了气,该把我粉身碎骨我也不皱眉。>
周妈妈:紫鹃,紫鹃姑娘,我也是上命差遣,概不由己啊。紫鹃,紫鹃,哎呀,好吧,就让雪雁跟我去吧。怎么,一个都不去啊?嘿,上面是不饶人的,走!
雪 雁:紫鹃姐姐,紫鹃姐姐。
紫 鹃:雪雁,雪雁……
林黛玉:你听……
紫 鹃:姑娘,没有什么声音。
林黛玉:<笙箫管笛耳边绕,一声声犹如断肠刀。他那里是花烛面前相对笑,我这里是长眠孤馆谁来吊?>
紫 鹃:姑娘,姑娘。
林黛玉:妹妹,我是不中用了。<多承你伴我月夕和花朝,几年来一同受煎熬。实指望与你并肩共欢笑,谁知道风雨无情草木凋,从今后你失群孤雁向谁靠?>
紫 鹃:姑娘!
林黛玉:<只怕是寒食清明,梦中把我姑娘叫。>
紫 鹃:姑娘,姑娘。
林黛玉:妹妹,我托你一件事。
紫 鹃:什么呀?
林黛玉:黛玉在此没有亲人,我的身子是干净的,你好歹叫他们送我回去。
紫 鹃:姑娘,你……
林黛玉:<我质本洁来还洁去,休将白骨埋污淖。>宝玉,宝玉!你好……
紫 鹃:姑娘,姑娘!姑娘……

金玉良缘
喜 娘:宝二爷,请用合欢酒。新二奶奶,请用合欢酒。恭喜老太太、太太,恭喜琏二奶奶。
贾 母:哈哈……
王熙凤:下面领赏去吧。
喜 娘:是。
贾宝玉:林妹妹,你身子好了没有?我们好久没有见面了,总算盼到了这一天。
王熙凤:哎,宝兄弟……
贾 母:宝玉啊,须要稳重懂礼貌。
贾宝玉:林妹妹,今天是从古到今,天上人间,<是第一件称心满意的事啊!我合不拢笑口将喜讯接,数遍了指头把佳期待。总算是东园桃树西园柳,今日移向一处栽。此生得娶你林妹妹,心如灯花并蕊开。往日病愁一笔勾,今后乐事无限美。从今后与你春日早起摘花戴,寒夜挑灯把谜猜。添香并立观书画,步影随月踏苍苔。从此后俏语娇音满室闻,如刀断水分不开。这真是银河虽阔总有渡,牛郎织女七夕会。>咦?方才只见雪雁,却为何不见紫鹃哪?
王熙凤:噢,紫鹃吗?她的生肖冲了,因此不来。
贾 母:对呀。
贾宝玉:噢,原来如此。林妹妹你盖着这个东西做什么呀,我们何必用这些俗套呢?
王熙凤:哎,宝兄弟,哦,呵呵呵呵,宝兄弟呀,<做新郎总该懂温柔,休惹得新娘气带羞。多生喜欢你少痴傻,随缘随分莫贪求。老祖宗都是为你好,你须懂得孝顺乃是第一筹。>
贾宝玉:你说我傻,哼,我看你才傻呢,我这个心都交给她了,还会对她不温柔么?<我爱她敬她都来不及,怎会使她气带羞?今日我十分喜减去一身病,百炼钢早化作绕指柔。>
王夫人:宝玉,我的儿,来,听娘说。<愿你俩相敬如宾到白头,这才是父母之心不辜负。>
贾宝玉:那还用说吗?
贾 母:宝玉。
贾宝玉:啊,
贾 母:来。
贾宝玉:噢。
贾 母:母亲的话可要记住哦。
王熙凤:是啊,宝兄弟。
贾宝玉:我知道。
贾 母:为你好啊。
贾宝玉:我知道。哎,林妹妹,虽然这红盖头,<遮住你面如芙蓉眉如柳,却遮不住你心底春光往外透。>嗳,但叫我如何不揭开它……啊!?
王夫人:宝玉啊!
贾 母:宝玉!
王夫人:宝玉啊,
贾 母:宝玉。
王夫人:宝玉你……
贾宝玉:啊呀,我在哪里呀,我在哪里呀?袭人,你来咬咬我的指头,看我是不是在做梦啊?
王熙凤:嗳,什么做梦不做梦,你不要胡说,老祖宗在这里坐着,老爷也在外面坐着呢。
贾宝玉:袭人,你告诉我,方才坐的那位美人儿是谁?
袭 人:新娶的二奶奶。
贾宝玉:哎呀,你真糊涂,新娶的二奶奶是谁?
袭 人:宝姑娘。
贾宝玉:林姑娘呢?
袭 人:啊呀,你怎么混说是林姑娘,老爷做主,娶的是宝姑娘。
王熙凤:是啊,是宝姑娘,谁说是林姑娘。
王夫人:宝玉,你娶的是宝姑娘。
王熙凤:是啊,是宝姑娘!
贾宝玉:宝姑娘,宝姑娘?老祖宗,你告诉我到底怎么样一回事,到底怎么样一回事?
贾 母:宝玉,你娶的是宝姑娘。
贾宝玉:是宝姑娘?我方才我明明同林妹妹成的亲,雪雁还扶着她呢!怎么,一霎时都变了,都变了?为什么,为什么?
王熙凤:宝兄弟,宝兄弟……老祖宗,船到桥头总会直的,你坐下,你坐下。
贾宝玉:林妹妹,林妹妹!
众 人:宝玉啊!宝兄弟!
贾宝玉:<我以为百年好事今宵定,为什么月老系错了红头绳?为什么梅园错把杏花栽,为什么鹊巢竟被鸠来侵。莫不是老祖宗骗我假做亲,宝姐姐她赶走我的心上人?>林妹妹,你在哪里?<定是气息奄奄你十分病,泪如滚水你煎着心。>听,你们听,你们可听到林妹妹的哭声,你们可听到林妹妹的哭声?
王夫人:宝玉啊,你静一静,听娘说,听娘说。啊……
贾宝玉:老祖宗,我要死了。
贾 母:宝玉啊,你怎么样,你怎么样啊?
贾宝玉:老祖宗,我有句心里的话要说……
贾 母:你讲,你讲,宝玉啊……
贾宝玉:<我和妹妹都有病,两个病源一条根,望求你把我们放在一间屋,也好让同病相怜心连心,活着也能日相见,死了也可葬同坟。>
贾 母:啊……
贾宝玉:<老祖宗啊,我天下万物我无所求,只求与妹妹共死生!>老祖宗,你就依了我吧,老祖宗。老祖宗,你就依了我吧……
王熙凤:宝兄弟……
王夫人:宝玉啊,你……
贾 母:宝玉啊!今天是你大喜之日,你竟病的这个样子,好叫我心痛啊!你……你呀!
贾宝玉:好,我知道,求你们也没有用。我找林妹妹去!
王熙凤:宝兄弟……
王夫人:宝玉。
贾宝玉:我找林妹妹去!
王夫人:宝玉……
贾 母:宝玉!你……
贾宝玉:我找林妹妹去!
薛宝钗:宝玉,林妹妹她已经死了。
贾宝玉:啊!?
王夫人:宝玉!
王熙凤:宝兄弟!
贾 母:宝玉!
合 唱:<好一条调包计偷柱换梁,只赢得惨红烛映照洞房。>

哭灵出走
贾宝玉:林妹妹,我来迟了……我来迟了!<金玉良缘将我骗,害妹妹魂归离恨天。到如今人面不知何处去,空留下素烛白帏伴灵前。林妹妹呀,林妹妹呀!如今是千呼万唤唤不归,上天入地难寻见。可叹我生不能临别话几句,死不能扶一扶七尺棺。林妹妹,想当初你是孤苦伶仃到我家来,只以为暖巢可栖孤零雁。我和你情深犹如亲兄妹,那时候两小无猜共枕眠。到后来我和妹妹都长大,共读《西厢》在花前。宝玉是剖腹掏心真情待,妹妹你心里早有你口不言。到如今无人共把《西厢》读,可怜我伤心不敢立花前。曾记得怡红院尝了闭门羹,你是日不安心夜不眠。妹妹呀,你为我是一往情深把病添,我为你是睡里梦里常想念。好容易盼到洞房花烛夜,总以为美满姻缘一线牵。想不到林妹妹变成宝姐姐,却原来你被逼死我被骗。实指望白头能偕恩和爱,谁知晓今日你黄土垅中独自眠。林妹妹啊,自从居住大观园,几年来你是心头愁结解不开。落花满地令你惊,冷雨敲窗你病未眠。你怕那人世上风刀和霜剑,到如今它果然逼你丧九泉。>
紫 鹃:宝二爷,天夜了,你不便多留,快回去吧。
贾宝玉:紫鹃,我知道妹妹恨我,你也恨我,我就是死了也是个屈死鬼。
紫 鹃:这些话我已经听惯了,人已死了还说什么呢?
贾宝玉:妹妹临死时,她讲些什么?
紫 鹃:唉。<想当初姑娘病重无人理,床前只有我知心婢。她这边是冷屋鬼火三更泣,你那边是洞房春暖一天喜。只听她恨声呼宝玉,这辛酸的事儿我牢牢记。>宝二爷,你来迟了,<你来迟了,人死黄泉难扶起。>
贾宝玉:林妹妹,你不能怪我,这是父母做主,并不是我负心!
紫 鹃:姑娘啊……
贾宝玉:<问紫鹃妹妹的诗稿今何在?>
紫 鹃:<如翩翩蝴蝶火中化。>
贾宝玉:<问紫鹃妹妹的瑶琴今何在?>
紫 鹃:<琴弦已断你休提它。>
贾宝玉:<问紫鹃妹妹的花锄今何在?>
紫 鹃:<花锄虽在谁葬花。>
贾宝玉:<问紫鹃妹妹的鹦哥今何在?>
紫 鹃:<那鹦哥叫着姑娘学着姑娘生前的话呀。>
贾宝玉:<那鹦哥也知情和义,>
紫 鹃:<世上的人儿不如它!>
贾宝玉:唉,林妹妹,我被人骗了,我被人骗了。<九州生铁铸大错,一根赤绳把终身误。天缺一角有女娲,心缺一块难再补。你已是无瑕白玉遭泥陷,我岂能一股清流随俗波!从今后啊,你长恨孤眠在地下,我怨种愁根永不拔。人间难栽连理枝,我和你世外去结并蒂花。>
紫 鹃:宝二爷,你快回去吧。
贾宝玉:是,回去吧,回去?
合 唱:<抛却了莫失莫忘通灵玉,挣脱了不离不弃黄金锁。离开了苍蝇竞血肮脏地,>
众 人:二爷……宝二爷……宝兄弟……你找到宝二爷没有……
合 唱:<撇开了黑蚁争穴富贵窠。>
%%>>>

\end{document}
