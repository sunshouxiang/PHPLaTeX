%%# xelatex -shell-escape %TEXFILE%
\documentclass{article}

\usepackage[font=opentype,pinyin=math,href=yes,titlesec=yes]{xepkgs}
\usepackage[left=2cm,right=2cm,top=2cm,bottom=2cm,dvipdfm]{geometry}
\usepackage{indentfirst}

%\titleformat{\section}{\heiti\huge}{\CJKnumber{\thesection}}{1em}{}
\titleformat{\section}{\centering\heiti\huge}{}{1em}{}
\renewcommand{\thefootnote}{\normalsize[\arabic{footnote}] }

%%! $note_spec = "footnote,color=darkgray,font=fangti,size=normalsize,raise=0.5ex";

\newcommand{\zyparx}[3]{{%
	\vspace{-3ex}%
	\setzysep{{\expandafter\csname #2\endcsname\ \tiny\ }}%
	\zypar*[pysize=normalsize,pycolorx=red,ccfont=#1,ccsize=#2]{footnote,color=darkgray,font=fangti,size=normalsize,raise=2ex}{#3}%
	\resetzysep%
}}%
\newcommand{\zyparxx}[3]{{%
	\vspace{-3ex}%
	\setzysep{{\expandafter\csname #2\endcsname\ \tiny\ }}%
	\zypar[pysize=normalsize,pycolorx=red,ccfont=#1,ccsize=#2]{footnote,color=darkgray,font=fangti,size=normalsize,raise=2ex}{#3}%
	\resetzysep%
}}%

%%! $title = '荷塘月色'; $author = '朱自清';

\begin{document}

%%=\section{{<? $title ?>} (\fangti {<?php echo $author;?>})}
\section{荷塘月色 (\fangti 朱自清)}%%=

%%<<<*hetang
%%! zhuyin($hetang[0]);
%%! zhuyin($hetang[1]);
%%! zhuyin($hetang[2]);
%%! zhuyin($hetang[3]);
%%! zhuyin($hetang[4]);
%%! zhuyin($hetang[5]);
%%! zhuyin($hetang[6]);
%%! zhuyin($hetang[7], 'fangti', 'LARGE', false, true);
%%! zhuyin($hetang[8], 'kaiti', 'huge', false, false);
%%! zhuyin($hetang[9]);
%%! zhuyin($hetang[10], 'fangti', 'LARGE', false, true);
%%! zhuyin($hetang[11]);
%%! zhuyin($hetang[12]);

\zyparxx{kaiti}{huge}{%%=
这几天心里颇不宁静。今晚在院子里坐着(zhe)乘凉,忽然想起日日走过的荷塘,
在这满月的光里,总该另有一(yi4)番样子(zi)吧(ba)。月亮(liang)渐渐地(de)升高了(le),墙外马路上(shang)
孩子(zi)们的欢笑,已经听不(bu2)见了(le);妻在屋里拍着闰儿(er)@{闰儿:指作者的次子朱闰生。},
迷迷(mi)糊(hu)糊(hu)地(de)哼着(zhe)眠歌。我悄悄地(de)披了(de)大衫,带上门出去(qu)。
}%%=

\zyparxx{kaiti}{huge}{%%=
沿着(zhe)荷塘,是一(yi4)条曲折的小煤屑路。这是一(yi4)条幽僻的路;白天也少人走,
夜晚更加寂寞。荷塘四周,长(zhang3)着许多树,蓊(weng3)蓊(weng3)郁郁@{蓊蓊郁郁:形容树木茂盛的样子。}的。
路的一(yi4)旁,是些(xie)杨柳,和一(yi4)些不知道名字(zi)的树。没有月光的晚上(shang),这路上(shang)
阴森森的,有些怕人。今晚却很好,虽然月光也还是淡淡的。
}%%=

\zyparxx{kaiti}{huge}{%%=
路上(shang)只(zhi3)我一(yi2)个(ge)人,背着(zhe)手踱着(zhe)。这一(yi2)片天地好像是我的;我也像超出了(le)平
常的自己,到了(le)另一(yi2)个(ge)世界里(li)。我爱热闹(nao),也爱冷静;爱群居,也爱独处(chu3)。
像今晚上(shang),一(yi2)个(ge)人在这苍茫的月下,什么都(dou)可以想,什么都(dou)可以不想,便
觉是个(ge)自由(you)的人。白天里(li)一(yi2)定要做的事,一(yi2)定要说的话,现在都(dou)可不理。
这是独处(chu3)的妙处,我且受用这无边的荷香月色好了(le)。
}%%=

\zyparxx{kaiti}{huge}{%%=
曲曲折折的荷塘上面(mian),弥望@{弥望:充满视野,满眼。}
的是田田@{田田:形容荷叶相连的样子。古乐府《江南曲》中有“荷叶何田田”的句子。}
的叶子(zi)。叶子(zi)出水很高,像亭亭的舞女的裙。层层的叶子(zi)
中间,零星地(de)点缀着(zhe)些(xie)白花,有袅娜(nuo2)@{袅娜:柔美的样子。}地(de)开
着(zhe)的(de),有羞涩地(de)打着(zhe)朵儿(er)的(de);正如一(yi2)粒粒的明珠,
又如碧天里的星星(xing),又如刚出浴的美人。微风过处,送来缕
缕清香,仿佛(fu2)远处高楼上渺茫的歌声似(shi4)的。这时候叶子(zi)与花
也有一(yi4)丝的颤动,像闪电般,霎时传过荷塘的那边去了(le)。叶
子(zi)本是肩并肩密密地(de)挨着(zhe),这便宛然@{宛然:仿佛。}有了(le)一(yi2)道凝碧的波痕。
叶子(zi)底下是脉(mo4)脉(mo4)@{默默地用眼神或行动表达情意。也用以形容水没有声音、好像深含感情的样子。}的
流水,遮住了(le),不能见一(yi4)些颜色;而叶子(zi)却更见风致@{风致:美的姿态。}了(le)。
}%%=

\zyparxx{kaiti}{huge}{%%=
月光如流水一(yi4)般,静静地(de)泻在(zai)这一(yi2)片叶子(zi)和花上(shang)。薄薄的青雾浮起在荷塘
里。叶子(zi)和花仿佛(fu2)在牛乳中洗过一(yi2)样;又像笼着(zhe)轻纱的梦。虽然是满月,
天上(shang)却有一(yi4)层淡淡的云,所以不能朗照;但我以为(wei2)这恰是到了(le)好处 —— 酣眠
固不可少,小睡也别有风味的。月光是隔了(le)树照过来的,高处丛生的灌木,
落下参(cen1)差(ci1)的斑驳@{斑驳:原指一种颜色中杂有别的颜色。这里有深浅不一的意思。也写作“班驳”。}
的黑影,峭楞楞如鬼一(yi4)般;弯弯的杨柳的稀疏的倩影@{倩影:美丽的影子。},却又像是画在荷叶上(shang)。
塘中的月色并不均匀;但光与影有着(zhe)和谐的旋律,如梵婀玲@{梵阿玲:英语“violin”的音译,即小提琴。}上(shang)奏着(zhe)的名曲(qu3)。
}%%=

\zyparxx{kaiti}{huge}{%%=
荷塘的四面,远远近近,高高低低都(dou)是树,而杨柳最多。这些树将一(yi2)片荷
塘重(chong2)重(chong2)围住;只(zhi3)在小路一(yi4)旁,漏着(zhe)几段空(kong4)隙,像是特为月光留下的。树色
一(yi2)例@{一例:一概,一律。}是阴阴的,乍看像一(yi4)团烟雾;但
杨柳的丰姿@{丰姿:风度,仪态,一般指美好的姿态。也写作“风姿”},
便在烟雾里也辨得(de)出。树梢上(shang)隐隐约约的是一(yi2)带远山,只(zhi3)有些(xie)大意罢了(le)。
树缝里也漏着(zhe)一(yi1)两点路灯光,没精打采的,是渴睡人的眼。这时候最热闹(nao)
的,要数(shu3)树上(shang)的蝉声与水里的蛙声;但热闹(nao)是它们(men)的,我什么也没有(you)。
}%%=

\zyparxx{kaiti}{huge}{%%=
忽然想起采莲的事情(qing)来了(le)。采莲是江南的旧俗,似乎很早就有,而六朝时
为(wei2)盛;从诗歌里可以约略知道。采莲的是少年的女子,她们(men)是荡着(zhe)小船,
唱着(zhe)艳歌@{艳歌:专门描写男女爱情的歌曲。}去的。采莲人不(bu2)用说
很多,还有看采莲的人。那是一(yi2)个(ge)热闹(nao)的季节,也是一(yi2)个风流@{风流:这里的意思是年轻男女不拘礼法地表露自己的爱情。}的
季节。梁元帝《采莲赋》里(li)说得(de)好:
}%%=

\noindent\begin{quote}\zyparxx{fangti}{LARGE}{%%=
于是妖童媛(yuan4)女,荡舟心许@{妖童媛女,荡舟心许:艳丽的少男和美貌的少女,摇着小船互相默默地传情。妖,艳丽。媛女,美女。 许,默认。};
鹢(yi4)首@{鹢首:古时画鹢于船头,所以把船头叫鹢首。鹢,水鸟。}徐回,
兼传羽杯;櫂(zhao4)@{櫂:通“棹”,划船的一种工具,形状和桨类似。}将移
而藻挂,船欲动而萍开。尔其纤腰束素,迁延顾步;夏始春余,叶嫩花初,恐沾裳(chang2)而浅笑,
畏倾船而敛裾(ju1)@{敛裾:这里是提着衣襟的意思。裾,衣襟。}。
}\end{quote}%%=

\noindent\zyparxx{kaiti}{huge}{%%=
可见当时嬉游的光景了(le)。这真是有趣的事,可惜我们(men)现在早已无福消受了(le)。
}%%=

\zyparxx{kaiti}{huge}{%%=
于是又记起,《西州曲(qu3)》@{《西洲曲》:南朝乐府诗,描写一个青年女子思念意中人的痛苦。}里的句子:
}%%=

\noindent\begin{quote}\zyparxx{fangti}{LARGE}{%%=
采莲南塘秋,莲花过人头;低头弄莲子,莲子清如水。
}\end{quote}%%=

\zyparxx{kaiti}{huge}{%%=
今晚若有采莲人,这儿(er)的莲花也算得(de)“过人头”了(le);只(zhi3)不(bu2)见一(yi4)些流水的影子(zi),
是不行的。这令我到底惦着(zhe)江南了(le)。 —— 这样想着(zhe),猛一(yi4)抬头,不觉已是自己的门前;
轻轻地(de)推门进去(qu),什么声息也没有,妻已睡熟好久了(le)。
}%%=

%%=\begin{flushright}一(yi1)九二七年七月,北京清华园。\end{flushright}
\begin{flushright}\zyparxx{kaiti}{huge}{%%=
一(yi1)九二七年七月,北京清华园。%%=
}\end{flushright}%%=

%%>>>

\end{document}

%%{
function zhuyin($text, $font = 'kaiti', $size = 'huge', $indent = true, $quote = false)
{
	global $note_spec;
	if ($quote) { echo "\\begin{quote}%\n"; }
	echo "\\zyparx{{$font}}{{$size}}{%\n";
	if (!$indent) echo "\\noindent ";
	$text = str_replace("一(yi", "一(-1.6ex/yi", $text);
	print_zy($text, false, $note_spec);
	echo "\n}\n";
	if ($quote) { echo "\\end{quote}\n"; }
	echo "\n";
}
%%}
