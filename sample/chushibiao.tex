%%! define('NOTE_STYLE', 'gezhu');

\documentclass[a4paper]{article}

\usepackage[font=opentype,pinyin=math,href=yes,titlesec=yes]{xepkgs}
\usepackage[left=2cm,right=2cm,top=2cm,bottom=2cm,dvipdfm]{geometry}
\usepackage{indentfirst}

\titleformat{\section}{\centering\heiti\huge}{}{1em}{}
\setlength{\parskip}{1.5\baselineskip}

\newcommand{\comment}[1]{{\noindent\kaiti\large #1}}
%%{
if (NOTE_STYLE == 'gezhu')
    $note_spec = "gezhu,color=darkgray,font=fangti";
else
    $note_spec = "footnote,color=darkgray,font=fangti,size=normalsize,raise=1ex";
echo "\\newcommand{\\zyparx}[1]{{\\vspace{-3ex}\\zypar*[pysize=normalsize,ccsize=huge]{{$note_spec}}{#1}}}\n";
%%}

\usepackage{footmisc}

\renewcommand{\thefootnote}{\normalsize[\arabic{footnote}] }

\begin{document}

\section[]{出师表(\fangti 诸葛亮)}

%%<<<*paras
%%! foreach ($paras as $par) {
%%!     if (substr($par, 0, 8) != '\\comment') {
%%!         echo "\zyparx{%\n";
%%!         print_zy($par, false, $note_spec);
%%!         echo "\n}\n";
%%!     } else {
%%!         echo $par, "\n";
%%!     }
%%!     echo "\n";
%%! }

%%%=\comment{\color{blue}《出师表》是一篇议论文。诸葛亮完美地遵循了议论文的写作结构:总论——分论——总论。
%%%=议论方法也用了举例论证、事实论证、道理论证、对比论证、演绎论证等等。}

臣亮言@{臣亮言:臣诸葛亮进言。《出师表》是三国时期(公元227年)蜀汉丞相诸葛亮在决定北上伐魏,夺取长安(今西安汉长安城遗址)之前给后主刘禅进的表章。出师:出兵。表:古代大臣给皇帝的陈述建议、意见或报告的文书,又叫“表章”、“奏章”}:
先帝@{先帝:指蜀汉昭烈帝刘备。因刘备此时已死,故称先帝。蜀汉只有刘备和刘禅(阿斗)两位皇帝,通常称刘禅为后主}创业@{创:开创,创立。业:消灭魏吴、统一天下、恢复汉朝的大业}未半@{未半:还不到一半,此处指没完成统一天下、兴复汉朝的大业},
而中道@{中道:中途,半路}崩殂@{崩殂:死。古代帝王死亡叫“崩”,“驾崩”。殂,死亡},
今@{今:现在}天下三分@{三分:东汉末年,中央政权腐败衰弱,诸侯割据,经过长期争斗、吞并,天下分为孙权(吴),曹操(魏),刘备(蜀)三大势力,并最终成为三个国家},
益州@{益州:汉代行政区域十三刺史部之一,包括今四川省和陕西省一带,这里指蜀汉政权}疲弊@{疲弊:困乏无力。关羽丢失荆州、刘备讨伐东吴失败后,蜀汉元气大伤,人力缺乏,物力缺无,民生凋敝。“益州疲弊”指蜀汉力量衰微,处境艰难},
此诚@{此:这。诚:的确,实在}危急存亡@{危急存亡:危急程度关系到蜀汉政权能否存在下去}之秋@{秋:时候,这里指关键时期,一般多指不好的(如“多事之秋”)}也。
然@{然:然而}侍卫之臣@{侍卫之臣:服侍、保卫皇帝的臣下}不懈@{懈:松懈,懈怠}于内@{内:皇宫},
忠志之士@{忠志之士:为国尽忠的仁人志士}忘身@{忘身:舍身忘死,奋不顾身}于外@{外:皇宫之外}者@{者:...的原因,犹如“之所以”},
盖@{盖:连词,表推断原因,相当于“应该是因为”}追@{追:追念,没有忘记}先帝之殊遇@{殊遇:优待,厚遇。殊,不一般,特殊},
欲报之于陛下也@{因为受到了先帝刘备的优遇,所以要向陛下您报恩}。
诚宜@{诚:的确,实在。宜:应该}开张圣听@{开张圣听:扩大主上的听闻,意思是要后主广泛听取别人的意见。开张,扩大,与下文“塞”相对},
以光先帝遗德@{光:发扬光大。遗德:留下的美德},
恢弘@{恢弘:发扬扩大。恢,大。弘,大,宽,这里是动词,也作“恢宏”}志士之气@{气:士气、志气、斗志},
不宜妄自菲(fei3)薄(bo2)@{妄自菲薄:过于看轻自己。妄,过分、随意。菲薄:微薄,小看,轻视},
引喻失义@{引喻失义:讲话不当。引喻:称引、比喻。失义:失当,违背大义},
以@{以:以致(与下文“以伤先帝之明”的“以”用法相同,意为“以致”)}塞(se4)@{塞:阻塞}忠谏@{谏:劝谏,指古代臣下对君主提建议和意见}之路也。

%%%=\comment{\color{blue} 总论点:治理好国家要听你爸爸的话,学习你爸爸那样治理国家。\newline
%%%=分论点1:要学习你爸爸扩大圣听,不要堵塞臣子忠言劝谏的道路。}

宫中府中@{宫:指皇宫。府:指丞相府},
俱为一体@{俱为一体:都属于一个整体。俱:通“具”,全,都},
陟罚臧否(pi3)@{陟:提升,奖励,晋升。罚:惩罚。臧否:善恶,这里用作动词,分别是“表扬”和“批评”的意思},
不宜异同@{异同:不同}。
若有作奸犯科@{作奸犯科:干不正当的事违犯法令。作奸:干坏事。科:科条,法令}及@{及:以及}为@{为:做}忠善@{忠善:忠君报国,积德行善}者@{者:...的人},
宜付有司论其刑赏@{宜付有司论其刑赏:应交给主管官吏,判定他们是应该受罚还是受奖。有司:官吏,这里指主管刑赏的官吏。论:判定},
以昭陛下平明之理@{昭:显示,表明。平:公平。明:严明。理:治理。整句的意思是:以此表明陛下的治理是公正严明的},
不宜偏私@{偏私:偏袒,有私心},
使内外异法也@{内外:指官廷内外。异法:行不同的法度,指赏罚不同。这几句话,据《三国志·蜀志·董允传》,可能是指刘禅偏袒宦官黄皓而讲的}。

%%%=\comment{\color{blue} 分论点2:治理好国家要赏罚严明,不偏不向,公正无私。\newline
%%%=道理论证:朝内朝外都是一样的,都是一体的,所以赏罚也需要有一样的标准。}

侍中、侍郎@{侍中、侍郎:官名,皇帝的亲臣。}郭攸之@{郭攸之:南阳人,当时任刘禅的侍中。}、费祎@{费祎:字文伟,江夏人,刘备时任太子舍人,刘禅继位后,任费门侍郎,后升为侍中。}、董允@{董允:字休昭,南郡枝江人,刘备时为太子舍人,刘禅继位,升任黄门侍郎,诸葛亮出师时又提升为侍中}等,
此皆良实@{良:善良,端正。实:踏实,诚实,靠得住},
志虑忠纯@{志:志向。虑:思想,心思。忠纯:忠诚纯正},
是以@{是以:因此}先帝简拔以遗(wei4)陛下@{简:挑选,一说通“拣”。拔:选拔,提升。遗:赠与,留给}。
愚@{愚:我,诸葛亮对自己的谦称}以为@{以为:认为}宫中之事,
事无大小,
悉以咨之@{悉:全部。咨之:征求郭攸之等人的意见。咨:询问,征求意见。之:指郭攸之等人},
然后施行,
必能裨补阙漏@{裨:补。阙漏:同“缺漏”,缺点和疏漏},
有所广益@{有所广益:有所启发和帮助。广益:增益,成语有“集思广益”。广:扩大。益:好处}。

将军向宠@{向宠:三国襄阳宜城人,刘备时任牙门将,刘禅继位,被封为都亭侯,后任中部督,即禁卫军统帅},
性行淑均@{性行淑均:性格品德善良端正。淑:善良。均:公正},
晓畅军事@{晓畅:精通,通晓},
试用于昔日@{试用于昔日:据《三国志·蜀志·向朗传》记载,章武二年(公元222年)刘备在秭归一带被东吴军队击败,而向宠的部队损失却很少,“试用于昔日”当指此事},
先帝称之曰能@{先帝称之曰能:先帝说他有能力},
是以众议举宠为督@{督:指中部督}。
愚以为营中之事@{营:军营、军队},
悉以咨之,
必能使行(hang2)阵和睦@{行阵:指部队。现在也有“行伍出身”这样的用法。古代五个士兵为一“伍”,好比现在部队的一个班},
优劣得所@{优劣得所:能力好坏各得其所,即用人得当,无论能力大小都安排了合适的工作}。

%%%=\comment{\color{blue} 事实论证:这几位都是先帝认可的对国家有帮助的大臣,所以你要重用。}

亲贤臣@{亲:亲近。贤臣:善良、正直、有能力的大臣},
远小人@{远:疏远。小人:晚辈,下人,奸人,这里指宦官(太监)},
此先汉@{先汉:前汉,即西汉}所以@{所以:之所以}兴隆@{兴隆:兴旺发达}也;
亲小人,
远贤臣,
此后汉@{后汉:东汉。西汉因为王莾篡位而灭亡。八年后光武帝刘秀消灭了王莾建立东汉,定都洛阳,史称光武中兴。因为洛阳在原来西汉都城长安以东,所以称为东汉}所以倾颓@{倾颓:倾覆,灭亡}也。
先帝在时,
每与臣论此事,
未尝不叹息痛恨于桓、灵@{桓、灵:指桓帝刘志、灵帝刘宏。这两个东汉末年的皇帝政治腐败,亲信宦官,使刘汉王朝倾覆}也。
侍中、尚书、长(zhang3)史、参军@{侍中:指郭攸之、费祎、董允等人。尚书:这里指陈震,南阳人,公元225年(建兴3年)任尚书,后升为尚书令。长史:这里指张裔,成都人,刘备时曾任巴湘乡人,当时任参军。诸葛亮出驻汉中,留下蒋琬、张裔统管丞相府事,后又暗中上奏给刘禅云“臣若不幸,后事宜以付琬”},
此悉贞良@{贞良:忠贞善良}死节@{死节:为国而死的气节,能够以死报国}之臣,
愿陛下亲之信之,
则汉室之隆@{隆:兴盛},
可计日而待@{计日:计算着天数,指为时不远}也。

%%%=\comment{\color{blue} 分论点3:亲贤远佞。\newline
%%%=举例论证:先汉的兴隆,后汉的衰落,还有桓帝、灵帝的例子用以证明此段分论点。\newline
%%%=对比论证:拿先汉兴隆的原因与后汉衰落的原因作对比,证明亲贤远佞的重要性、正确性。}

臣本布衣@{布衣:平民,百姓},
躬耕于南阳@{躬耕:自己耕种。南阳:指隆中,在湖北省襄阳城西,当时隆中属南阳郡管辖},
苟全@{苟:苟且;全:保全}性命于乱世,
不求闻达于诸侯@{闻:有名望,闻名。达:通达,此指扬名显贵。诸侯:这里指当时割据一方的军阀}。
先帝不以臣卑鄙@{卑鄙:地位、身分卑下,见识粗鄙。卑,身分低下。鄙,粗鄙,粗野},
猥自枉屈@{猥:屈辱。枉屈:枉驾屈就。诸葛亮认为刘备三顾茅庐去请他,对刘备来说是屈辱,自己不该受到刘备亲自登门拜请的待遇。这是一种客气的说法},
三顾臣于草庐之中@{三顾臣于草庐之中:即刘备三次去隆中请诸葛亮之事。顾,看,看望,拜望},
咨臣以当世之事@{当世之事:指天下形势和创业大计},由是@{由是:由此,因此}感激,遂(sui4)@{遂:于是,因而}许先帝以驱驰@{许:答应。驱驰:指奔走效力}。
后值倾覆@{后值倾覆:之后遇到危难。建安十三年(公元208年)刘备在当阳长坂坡被曹操打败,退至夏口,派诸葛亮去联结孙权,共同抵抗曹操。本句,连同下句即指此事},
受任于败军之际,
奉命于危难(nan4)之间,
尔来@{尔来:从那时以来。即从刘备三顾茅庐到诸葛亮出师北伐以来}二十有(you4)@{有:通“又”}一年矣。

%%%=\comment{\color{blue} 演绎论证:铺陈自身经历,看似与主论点无关,但是其实还是在申明主论点,告诫后主要向先帝学习:礼贤下士,开张言路,盖以呼应首段“然侍卫之臣不懈于内,忠志之士忘身于外者,盖追先帝之殊遇,欲报之于陛下也。”}

先帝知臣谨慎,
故临@{临:将要,临近}崩寄@{寄:托付}臣以大事也@{大事:指章武三年(公元223年)刘备临终前嘱托诸葛亮辅佐刘禅,兴复汉室,统一天下的大事。刘备临终时对刘禅说“汝与丞相从事,事之如父。”}。
受命以来,
夙夜@{夙夜:日夜。夙,清晨}忧叹,
恐@{恐:惟恐,担心}托付不效@{不效:没有成效},
以伤先帝之明@{以伤先帝之明:以致有损先帝的英明,指如果托付的事业没有完成,就等于先帝看错了人},
故五月渡泸@{五月渡泸:建兴元年(公元223年)云南少数民族的上层统治者发动叛乱,建兴三年(公元225年)诸葛亮率军南征,在五月渡过泸水,到秋天平定了这次叛乱(七擒孟获),下句“南方已定”即指此事。泸:泸水,即金沙江},
深入不毛@{不毛:不长草木,这里指人烟稀少的地方。毛:庄稼,禾苗}。
今南方已定,
兵甲@{兵:武器。甲:装备}已足,
当奖率(shuai4)三军@{奖率:激励率领。三军:古代诸侯国的军队分上、中、下三军,三军即全军},
北定中原,
庶竭驽钝@{庶:庶几,希望。竭:竭尽。驽钝:比喻自己的低劣的才能。驽:劣马,指才能低劣。钝:刀刃不锋利,指头脑不灵活,做事愚钝。这是诸葛亮自谦的说法},
攘除奸凶@{攘除:排除,铲除。奸凶:奸邪凶恶之人,此指曹魏政权},
兴复汉室,
还于旧都@{旧都:指东汉都城洛阳}。
此臣所以报先帝而忠陛下之职分(fen4)@{这是我用来报答先帝,效忠陛下的职责本分。职分:应尽之责}也。
至于斟酌损益@{斟酌损益:斟情酌理、有所兴革。斟酌:权衡,有分寸;损:除去;益:兴办,增加},
进尽忠言,
则攸之、祎、允之任也。

愿陛下托臣以讨贼兴复之效@{托臣以讨贼复之效:把讨伐曹魏复兴汉室的任务交给我。托:委托,交给。效:效命的任务},
不效@{不效:没有成效}则治臣之罪,
以告@{告:告慰,告祭}先帝之灵。
若无兴德之言@{兴德之言:发扬陛下恩德的忠言},
则责攸之、祎、允等之慢@{慢:怠慢,疏忽,指不尽职},
以彰其咎@{彰:表明,彰显。咎:过失,罪过};
陛下亦宜自谋@{亦宜自谋:自己也应该多考虑事情},
以谘诹善道@{谘诹善道:征求(治国的)良策。诹:询问,咨询。善:好。道:方法,策略},
察纳雅言@{察纳:识别采纳。察:明察,辨别。雅言:正确的言论,合理的意见},
深追先帝遗诏@{深追;深刻地追念、追随、遵守。遗诏:皇帝在临终时所发的诏令。刘备临死时曾对刘禅说“勿以恶小而为之,勿以善小而不为”}。
臣不胜(sheng1)受恩感激。

今当@{当:在...的时候}远离,临表涕零@{临表涕零:面对着这份表章落泪。涕零:落泪},不知所言@{不知所言:不知道还要说些什么}。

%%%=\comment{\color{blue} 总论点重申:治理好国家要听你爸爸的话,学习你爸爸那样治理国家。}

%%%=\comment{\color{red} 饱含感情的议论文,你见过吗?这才是《出师表》真正的不朽之处!议论文好写,但字字含情的议论文,千年才出一篇!}

%%>>>

\end{document}
